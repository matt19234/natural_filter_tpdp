% \documentclass[10pt, twocolumn]{article}
\documentclass[10pt, letterpaper]{article}
\usepackage[margin=1in]{geometry}
\usepackage{setspace}
\setstretch{1.0} % change line spacing

\usepackage{dsfont}
\usepackage{amsmath}
\usepackage{amsthm}
\usepackage{amssymb}
\usepackage{amsfonts}
\usepackage{url}
\usepackage{enumerate}
\usepackage{subcaption}
\usepackage{bm}
\usepackage{graphicx}
\usepackage[utf8]{inputenc} % allow utf-8 input
\usepackage[T1]{fontenc}    % use 8-bit T1 fonts
\usepackage{hyperref}       % hyperlinks
\usepackage{booktabs}       % professional-quality tables
\usepackage{nicefrac}       % compact symbols for 1/2, etc.
\usepackage{microtype}      % microtypography
\usepackage{xcolor}         % colors
\usepackage{hyperref}
\usepackage{algorithm}
\usepackage{algorithmic}
\usepackage{cleveref}

\usepackage{tikz}
\usetikzlibrary{calc}
\usetikzlibrary{arrows.meta}

% \newcommand{\conv}{\textsc{conv}}
\newcommand{\conv}{\operatorname{conv}}
\newcommand{\compose}{\operatorname{compose}}
\newcommand{\fid}{f_{\operatorname{id}}}

\newcommand{\pld}{\operatorname{PLD}}
\newcommand{\Adv}{\operatorname{Adv}}
\newcommand{\Filter}{\operatorname{Filter}}
\renewcommand{\epsilon}{\varepsilon}

\newcommand{\algcomment}[1]{\hfill{$\triangleright$~{\em #1}}}

\newtheorem{lemma}{Lemma}
\newtheorem{theorem}{Theorem}
\newtheorem{corollary}{Corollary}
\newtheorem{definition}{Definition}
\newtheorem{proposition}{Proposition}
\newtheorem{remark}[theorem]{Remark}

\newcommand{\datasets}{\mathcal{D}}
\newcommand{\dataset}{D}
\newcommand{\mechanism}{\mathcal{M}}
\newcommand{\hypothesis}{\mathrm{H}}

\newcommand{\1}{\mathds{1}}
\newcommand{\ind}{\perp\!\!\!\!\perp} 
\newcommand{\E}{\mathop{\mathbb{E}}}
\newcommand{\mP}{\mathop{\mathbb{P}}}

\newcommand{\bE}{\mathbb{E}}
\newcommand{\bN}{\mathbb{N}}
\newcommand{\bP}{\mathbb{P}}
\newcommand{\bR}{\mathbb{R}}
\newcommand{\bZ}{\mathbb{Z}}

\newcommand{\cA}{\mathcal{A}}
\newcommand{\cB}{\mathcal{B}}
\newcommand{\cC}{\mathcal{C}}
\newcommand{\cF}{\mathcal{F}}
\newcommand{\cL}{\mathcal{L}}
\newcommand{\cG}{\mathcal{G}}
\newcommand{\cH}{\mathcal{H}}
\newcommand{\cM}{\mathcal{M}}
\newcommand{\cS}{\mathcal{S}}
\newcommand{\cT}{\mathcal{T}}
\newcommand{\cY}{\mathcal{Y}}
\newcommand{\cX}{\mathcal{X}}

\newtheoremstyle{TheoremNum}
{\topsep}{\topsep}              %%% space between body and thm
{\itshape}                      %%% Thm body font
{}                              %%% Indent amount (empty = no indent)
{\bfseries}                     %%% Thm head font
{.}                             %%% Punctuation after thm head
{ }                             %%% Space after thm head
{\thmname{#1}\thmnote{ \bfseries #3}}%%% Thm head spec
\theoremstyle{TheoremNum}
\newtheorem{numberedprop}{Proposition}
\newtheorem{numberedlemma}{Lemma}

\DeclareMathOperator{\supp}{supp}

\newcommand{\eps}{\epsilon}
\newcommand{\bp}{\mathbf{p}}
\newcommand{\beps}{\boldsymbol{\eps}}
\newcommand{\bell}{\boldsymbol{\ell}}
\newcommand{\R}{\mathbb{R}}
\newcommand{\N}{\mathbb{N}}
\newcommand{\cD}{\mathcal{D}}
\newcommand{\RRd}{\RR^{\downarrow}}
\newcommand{\adv}{\mathcal{A}}
\newcommand{\mech}{\mathcal{M}}
\newcommand{\nullmech}{\text{NULL}}
\newcommand{\Iddist}{\text{Id}}
\newcommand{\postproc}{\mathrm{Proc}}
\newcommand{\bj}{\mathbf{j}}
\newcommand{\bzero}{\mathbf{0}}
\newcommand{\bone}{\mathbf{1}}
\newcommand{\bt}{\mathbf{t}}
\newcommand{\tX}{\tile{X}}
\newcommand{\cO}{\mathcal{O}}
\newcommand{\cZ}{\mathcal{Z}}
\newcommand{\Ber}{\mathrm{Ber}}
\newcommand{\Var}{\mathrm{Var}}
\def \Paren#1{{\left({#1}\right)}}
\def \Brack#1{{\left[{#1}\right]}}
\def \Curly#1{{\left\{{#1}\right\}}}
\newcommand{\tdelta}{\tilde{\delta}}
\newcommand{\bb}{\mathbf{b}}


\newcommand{\todo}[1]{\textcolor{red}{[TODO] #1}}
\newcommand{\ml}[1]{\textcolor{purple}{ML: #1}}
\newcommand{\pierre}[1]{\textcolor{teal}{Pierre: #1}}
\newcommand{\matt}[1]{\textcolor{brown}{Matt: #1}}
\newcommand{\pasin}[1]{\textcolor{blue}{Pasin: #1}}
\newcommand{\ethan}[1]{\textcolor{violet}{Ethan: #1}}
\newcommand{\gk}[1]{\textcolor{orange}{Gautam: #1}}

\allowdisplaybreaks


% \title{Natural Privacy Filters Are Not Always Free: A Characterization of Free Natural Filters and the Cost of Adaptivity\thanks{Authors BH, EL, PM, and PT in alphabetical order.}}
\title{Natural Privacy Filters Are Not Always Free: A Characterization of Free Natural Filters\thanks{Authors BH, EL, PM, and PT in alphabetical order.}}

\author{
	Matthew Regehr\thanks{University of Waterloo. \texttt{matt.regehr@uwaterloo.ca}. Supported by an NSERC CGS-D scholarship.} \and
	Bingshan Hu\thanks{University of British Columbia. \texttt{bingsha1@cs.ubc.ca}} \and
	Ethan Leeman\thanks{Google Research. \texttt{ethanleeman@google.com}} \and
	Pasin Manurangsi\thanks{Google Research. \texttt{pasin@google.com}} \and
	Pierre Tholoniat\thanks{Google. \texttt{pierre@cs.columbia.edu}. Work done while at Columbia University.} \and
	Mathias L\'ecuyer\thanks{University of British Columbia. \texttt{mathias.lecuyer@ubc.ca}}
}

\begin{document}
	\maketitle

	\begin{abstract}
		We study natural privacy filters, which enable the exact composition of differentially private (DP) mechanisms with adaptively chosen privacy characteristics. Earlier privacy filters consider only simple privacy parameters such as R\'enyi-DP or Gaussian DP parameters. Natural filters account for the entire privacy profile of every query, promising more efficient use a given privacy budget. We show that, contrary to other forms of DP, natural privacy filters are not free in general. We present a characterization of when a family of private queries admits free natural filters for a given budget. We show that only families of privacy mechanisms that are well-ordered when composed admit free natural privacy filters with respect to an arbitrary privacy budget.
		% For other families of privacy mechanisms, we give the first XXX-bound for the cost of adaptivity under PLD composition.
	\end{abstract}

	\section{Introduction}

Differential Privacy (DP) \cite{dwork2006calibrating} is a rigorous notion of data privacy, used to protect against a host of different privacy attacks, including membership inference \cite{homer2008resolving,dwork2015robust,carlini2022membership} and data reconstruction \cite{dinur2003revealing,dwork2007price,carlini2021extracting,balle2022reconstructing,dick2023confidence,nasr2023scalable}.
As such, there is a sustained push to integrate and deploy DP in a large number of applications \cite{opendp-registry}, from machine learning (ML) and artificial intelligence (AI) models \cite{kairouz2021practical,sinha2025vaultgemma}, to analytics collection \cite{rogers2020linkedin,apple-emoji,ding2017collecting}, to data releases \cite{adeleye2023publishing,abowd20222020}, as well as in general libraries and systems supporting those applications \cite{gaboardi2020programming,wilson2020differentially,jax-privacy2022github,tholoniat2024cookie,ghazi2025differential}. % latter is systems and DB work, ad measurement, OpenDP, Google DP, libraries, ...
%
This practical DP push also motivates important theoretical progress.

A prominent direction of theoretical improvement comes from alternative definitions of DP, that enable tighter privacy loss accounting under composition and improve the privacy-accuracy tradeoff. The most successful definitions include $(\epsilon, \delta)$-DP \cite{dwork2006our} (known as approximate DP), the R\'enyi divergence family of DP definitions such as zCDP \cite{dwork2016concentrated,bun2016concentrated} and RDP \cite{mironov2017renyi}, and privacy definitions that generalize approximate DP by considering all $(\epsilon, \delta)$-DP values compatible with a given mechanism, such as the privacy loss distribution (PLD) \cite{sommer2019privacy} and $f$-DP \cite{dong2022gaussian}.
The latter definitions, PLD and $f$-DP, are theoretically appealing as they tightly capture resilience to membership inference attacks framed as hypothesis tests \cite{WassermanZ10,dong2022gaussian}. However, composition under these definitions is more complex and can be computationally difficult, with analyses of specific algorithms that rely on numerical approaches to handle composition in practice \cite{koskela2020computing,gopi2021numerical}.

Two other theoretical improvements enable new forms of composition that are necessary requirements for practical, real world DP usage.
First, concurrent composition \cite{vadhan2021concurrent} supports composition of DP mechanisms running over multiple rounds of interaction with the user---such as the sparse vector technique \cite{dwork2009complexity}---that are running in parallel.
This is required in practical systems which typically have to handle queries from multiple parties simultaneously, and where it is not possible to enforce a sequential ordering of the queries. Even individual users may want to run DP computations concurrently with long-running, multi-step mechanism (e.g., \cite{kostopoulou2023turbo}).

Second, adaptive composition, the main topic of this paper, enables the composition of DP mechanisms with adaptively chosen privacy parameters (as opposed to regular composition, in which these parameters have to be known in advance).
Adaptive composition is formalized through the notion of \emph{privacy filters} \cite{RRUV16}, which accept DP mechanisms as long as their composition remains within a bound fixed in advance  (more details in \S\ref{sec:preliminaries}).
Such adaptive composition is necessary to support long running applications, in which analysts can adaptively query a system with mechanisms and DP parameters of their choice \cite{lecuyer2019privacy,kostopoulou2023turbo,kuchler2024cohere}.
Furthermore, filters enable crucial privacy accounting savings through advanced versions of parallel composition \cite{lecuyer2019privacy}, as well as individual (personalized) DP \cite{ebadi2015differential}, which enables data dependent privacy accounting in ML model training \cite{feldman2021individual,yu2022individual} and measurement systems \cite{tholoniat2024cookie,ghazi2025differential}.

A key question in DP composition is whether these various improvements combine with each other.
%
Ideally these more flexible forms of composition come for free: that is, that can we apply composition results under concurrent mechanisms or adaptive budgets using privacy filters, without a penalty in the final privacy loss after composition.
Flexible composition for free is important not only because it is efficient privacy-wise, but also because it enables reusing all existing composition results, including tight analyses for specific mechanisms and empirical computational results.

As it turns out, concurrent composition is indeed free for all DP definitions \cite{vadhan2023concurrent,lyu2022composition}.
Moreover, concurrent and adaptive filter composition reduces to adaptive filters \cite{haney2023concurrent}, and thus has the same composition cost.
%
Therefore, the full expressivity of both concurrency and adaptivity hinges on free privacy filter composition for different notions of DP.
%
Filters, however, are not as well understood.
We know that adaptive composition under privacy filters is free for $\epsilon$-DP \cite{RRUV16} and R\'enyi-based DP definitions \cite{FZ21,lecuyer2021practical} (zCDP, RDP). Filters are also free for Gaussian mechanisms under $f$-DP and PLD accounting \cite{ST22,KTH22}. However, for $(\epsilon, \delta)$-DP, the best known result for adaptive composition with filters is only asymptotically free, with worse constants than non-adaptive composition \cite{WRRW23}. Even less is known about filters under $f$-DP and PLD. In particular, we consider \emph{natural filters}, which leverage lossless composition of the privacy characteristics of individual queries. Natural filters can be realized via PLDs, privacy profiles, or $f$-DP accounting.
In this paper, we make three important contributions towards fully understanding natural privacy filters:
\begin{itemize}
	\item First, in \S\ref{sec:adversary_theory} we present a general theory of adaptive mechanisms that issue adversarial queries with a constrained per-query privacy cost.
	\item In \S\ref{sec:natural_filter_characterization} we apply this theory to the natural filter and, surprisingly, we show that adaptive composition is NOT free under the PLD nor $f$-DP notions of privacy (see \Cref{corollary:no-free-filters}).
	\item Finally, we show in \S\ref{sec:well_ordering_proof} that filters are only free for very restricted families of mechanisms: those that are closed under composition, and admit totally ordered tradeoff functions.
	% \item Families admitting free filters are XXX (KL characterization) (\S\ref{}).
	% \item We then quantify the cost of adaptivity, showing for the first time an XXX-bound on the privacy cost of adaptive composition for PLD (\S\ref{}).
\end{itemize}

Our work also includes some minor contributions including that PLDs can be assigned a natural ordering satisfying a completeness property enabling one to take a supremum of PLDs (see \Cref{prop:sup-conv}). As far as we are aware this property has not been documented in the privacy accounting literature. We alo document in \S\ref{sec:implications-special-case} some families of queries and budgets that do and do not admit free filters. 

	\section{Preliminaries}
\label{sec:preliminaries}

A DP mechanism is randomized algorithm $M : \cD \to \cY$ where $\cD$ is some space of datasets containing sensitive information. In our work, the particular neighbouring relation over $\cD$ turns out to not be relevant. In fact, our conclusions hold for an arbitrary pair of datasets. For this reason, we fix throughout a single arbitrary pair of datasets $(D_1, D_2) \in \cD \times \cD$ and view the privacy characteristics of a mechanism $M$ in terms of the hypothesis testing problem $M(D_1)$ vs. $M(D_2)$ \cite{WassermanZ10,DRS22}.

In our work we reason about the composition of adaptive mechanisms, which are permitted to inspect outcomes of any previously run mechanism. It is enough to define adaptivity for just two rounds of interaction as we can define more complex adaptively composed mechanisms inductively.

\begin{definition}
	Let $M_1 : \cD \to \cY_1$ be a mechanism and let $M_2 : \cD \times \cY_1 \to \cY_2$ be an adaptive mechanism. The adaptive composition of $M_1$ and $M_2$ is the mechanism
	\begin{align*}
		(M_1 \otimes M_2)(D) := (Y_1, Y_2), Y_1 \sim M_1(D), Y_2 \sim M_2(D; Y_1).
	\end{align*}
\end{definition}

It is desirable to furnish hypothesis testing problems $(P, Q)$ with some kind of ordering so that we can compare the privacy guarantees offered by various mechanisms. We also require an ordering so that we can make sense of natural privacy filters. Fortunately testing pairs can be equipped with a natural information-theoretic order, namely the Blackwell order.

\begin{definition}
	For a pair of distributions $(P, Q)$ on a common probability space $\Omega$ and a second pair $(P', Q')$ on $\Omega'$, we say that $(P', Q')$ dominates $(P, Q)$ in the Blackwell order, written $(P, Q) \preceq (P', Q')$, if we can find a Markov kernel $\phi$ from $\Omega'$ to $\Omega$ such that $P = \phi P'$ and $Q = \phi Q'$.
\end{definition}

% Informally, the Blackwell order says that more information is available for distinguishing $P'$ and $Q'$ compared to $P$ vs $Q$. For mechanisms, we can interpret this as $(M(D_1), M(D_2)) \preceq (M'(D_1), M'(D_2))$ when $M$ can be reconstructed from $M'$ by introducing additional randomness through the kernel $\phi$.

We can consolidate hypothesis testing pairs into a privacy loss distribution (PLD) while still fully capturing privacy characteristics \cite{DworkR16}.
% To formalize the PLD for $(P, Q)$, we require the likelihood ratio (Radon--Nikodym derivative) $\frac{dP}{dQ}$. Formally speaking, this requires absolute continuity. However, by extending likelihood ratios with $\infty$ as needed, we can allow mechanisms that violate absolute continuity such as the $(\eps, \delta)$-RR mechanism.

\begin{definition}
	Let $P$ and $Q$ be distributions with likelihood ratio $\frac{dP}{dQ}$. The PLD of $(P, Q)$, denoted $\pld(P \parallel Q)$, is the distribution over $\bR \cup \{\infty\}$ of $\log(\frac{dP}{dQ}(\omega)), \omega \sim P$.
\end{definition}

For shorthand, we denote by $\pld(M) := \pld(M(D_1) \parallel M(D_2))$ the PLD of a mechanism $M$. PLDs have a remarkable algebraic structure: Letting $\Iddist$ denote the identically zero distribution and letting $\pld(P \parallel Q) \preceq \pld(P' \parallel Q')$ when $(P, Q) \preceq (P', Q')$, it turns out that PLDs form a partially-ordered commutative monoid under the convolution operator $\oplus$ and the Blackwell order $\preceq$. More remarkable still is that, like the real numbers, one can take define the supremum for PLDs. To the best of our knowledge this property has not been documented in the literature.

\begin{proposition}\label{prop:pld_sup}
    For any non-empty family of PLDs $\mathcal{L}$ bounded above by at least one PLD, there exists a unique PLD $\sup \mathcal{L}$ that dominates $\mathcal{L}$ but is dominated by every upper bound for $\mathcal{L}$.
\end{proposition}

	\section{Query-Restricted Adversaries}
\label{sec:adversary_theory}

In this section we develop a general theory of the privacy characteristics of adversarial adaptive mechanism when the queries the adversary may issue are restricted. This will lead directly to a theory of natural filters as we will see in \S\ref{sec:natural_filter_characterization}. For simplicity, we will assume that an adversary may always ``pass'' its turn by issuing a query with no privacy loss.

\begin{definition}
	Let $k$ be a positive integer. We define a privacy rule of length $k$ as a map $\Gamma$ that takes a sequence of PLDs $(L_1, \dots, L_{k'})$ ($0 \leq k' < k$) and outputs some collection of PLDs including $\Iddist$. We define a $\Gamma$-adversary as an adaptively composed mechanism $M = M_1 \otimes \dots \otimes M_k$ such that
	\begin{align*}
		L_i := \pld(M_i(\cdot; y_1, \dots, y_{i - 1})) \in \Gamma(L_1, \dots, L_{i - 1})
	\end{align*}
	for every $(y_1, \dots, y_{k - 1}) \in \cY_1 \times \dots \times \cY_{k - 1}$.
	% \matt{can relax to with probability 1 over (Y_i)_i}
	We will write $\Adv(\Gamma)$ to denote the set of $\Gamma$-adversaries and we will denote by
	\begin{align*}
		\pld(\Gamma) := \sup\{\pld(M) : M \in \Adv(\Gamma)\}
	\end{align*}
	the worst-case privacy loss of arbitrary $\Gamma$-adversaries.
\end{definition}

Crucially, notice that that no single adversary fully realizes the privacy cost of the rule $\Gamma$. Recalling the $f$-DP interpretation of the supremum (see \Cref{prop:sup-conv}), the privacy cost of the rule $\Gamma$ is in fact realized by a family of adversaries that minimize the false negative rate for every given confidence level. It turns out that the maximum privacy cost of the rule $\Gamma$ for a given confidence level occurs when the adversary first plays the best move for this confidence level, observes the outcome, then depending on the likelihood of having observed this outcome recursively chooses its remaining strategy tuned to maximize the overall privacy cost. Note that by $\lambda$ we mean the empty sequence.

\begin{lemma}
	\label{lemma:restricted_adversary}
	Let $\Gamma$ be a rule of length $k$. If $k = 1$, then $\pld(\Gamma) = \sup{\Gamma(\lambda)}$ and, for $k > 1$, we have
	\begin{align*}
		\pld(\Gamma) = \sup\{L \oplus \pld(\Gamma_L) : L \in \Gamma(\lambda)\}
	\end{align*}
	where $\Gamma_{L_0}(L_1, \dots, L_{k'}) := \Gamma(L_0, L_1, \dots L_{k'})$ denotes the rule of length of $k - 1$ that fixes the first move played against $\Gamma$ to $L_0$.
\end{lemma}

We prove the result by passing to hockey-stick curves because a pointwise supremum is easier to work with rather than the lower convex envelope. We will essentially show that a worst-case adversary first plays an optimal move with corresponding PLD $L$, observes the outcome $Y_1$, and adversarially chooses a followup in $\Adv(\Gamma_L)$ to ensure that the overall process ``traces out'' the hockey-stick curve of $L \oplus \Adv(\Gamma_L)$.

\begin{proof}
	The case where $k = 1$ is immediate, so we assume $k > 1$. We first show that
	\begin{align*}
		\pld(\Gamma) \preceq \sup\{L \oplus \pld(\Gamma_L) : L \in \Gamma(\lambda)\}.
	\end{align*}
	To that end, consider any $\Gamma$-adversary $M$. Decompose $M$ as the adaptive composition of $M_1(\cdot)$ with some $M'(\cdot; y_1)$ and let $L := \pld(M_1) \in \Gamma(\lambda)$. Then, clearly, for every fixed $y_1$, $M'(\cdot; y_1)$ is a $\Gamma_L$-adversary and, in particular, $\pld(M'(\cdot; y_1)) \preceq \pld(\Gamma_L)$. By \Cref{prop:composition_convolution}, we have that $\pld(M) \preceq L \otimes \pld(\Gamma_L)$ and the desired inequality follows by taking suprema.

	More surprising is the reverse inequality, which we prove constructively by passing to hockey-stick curves. Let $L \in \Gamma(\lambda)$, let $x \in \bR^\times$, and let $\gamma > 0$. Choose any mechanism $M_1$ that has PLD $L$ and let $\ell_1 := \frac{dM_1(D_1)}{dM_1(D_2)}$ denote the corresponding likelihood function. Moreover, for any $x \in \R^\times$ we can choose by \Cref{prop:sup-conv}, a $\Gamma_L$-adversary $M^x$ such that
	\begin{align*}
		H_x(M^x) \geq \sup_{M' \in \Adv(\Gamma_L)}H_x(M') - \gamma.
	\end{align*}
	Now consider the $\Gamma$-adversary $M := M_1 \otimes M_2$ where $M_2(D; y_1) := M^{x/\ell_1(y_1)}(D)$. By \Cref{prop:hockey_stick_composition}, we get
	\begin{align*}
		h_M(x)
			& = \E_{Y_1 \sim M_1(D_1)}[h_{M_{x/\ell_1(Y_1)}}(x/\ell_1(Y_1))] \\
			& \geq \E_{Y_1 \sim M_1(D_1)}[\sup_{M' \in \Adv(\Gamma_L)}h_{M'}(x/\ell_1(Y_1)) - \gamma] \\
			& = (h_{M_1} \otimes \sup_{M' \in \Adv(\Gamma_L)} h_{M'})(x) - \gamma.
	\end{align*}
	Again, taking suprema, we get that
	\begin{align*}
		\sup_{M \in \Adv(\Gamma)} h_M \succeq h_{M_1} \otimes \sup_{M' \in \Adv(\Gamma_L)} h_{M'},
	\end{align*}
	and thus $\pld(\Gamma) \succeq L \otimes \pld(\Gamma_L)$, which completes the proof since $L$ was arbitrary.
\end{proof}

	\section{Characterizing the Natural Filter}
\label{sec:natural_filter_characterization}

Our goal is to study when the natural filter \Cref{alg:natural_filter} with budget $B$ and queries in $\cL$ comes at no additional privacy cost, i.e. the mechanism induced by the interaction between the filter and an adversarial analyst itself has privacy bounded by $B$. In this case, we say that the natural filter is free. For simplicity, we will assume that $\Iddist \in \cL$. We also denote by $\cL^\infty$ the closure of $\cL$ under convolution.

\begin{algorithm}[tb]
	\caption{NaturalFilter}
	\label{alg:natural_filter}
	\begin{algorithmic}
		\REQUIRE Privacy budget $B$, analyst $\cA$, family of allowed PLDs $\cL$, query capacity $k$, dataset $D$
		\FOR{$i = 1, \dots, k$}
			\STATE $\cA$ gives mechanism $\cM_i$ that has PLD $L_i \in \cL$ \algcomment{can depend on previous results $Y_{<i}$}
			\IF{$L_1 \oplus \dots \oplus L_i \preceq B$ and no $Y_j = \bot, j < i$}
				\STATE $\cA$ receives $Y_i \sim \cM_i(D)$
			\ELSE
				\STATE $\cA$ receives $Y_i = \bot$
			\ENDIF
		\ENDFOR
		\RETURN $(Y_1, \ldots, Y_k)$
	\end{algorithmic}
\end{algorithm}

Now, notice that the natural filter \Cref{alg:natural_filter} with budget $B$ and $k$ queries in $\cL$ is free exactly when $\pld(\Filter_{\cL, B, k}) \preceq B$ where
\begin{align*}
	\Filter_{\cL, B, k}(L_1, \dots, L_{k'})
		:= \{L_{k' + 1} \in \cL : L_1 \oplus \dots \oplus L_{k'} \oplus L_{k' + 1} \preceq B\}
\end{align*}
for $0 \leq k' < k$. In particular, by applying \Cref{lemma:restricted_adversary} to the rule $\Filter_{\cL, B, k}$ we immediately get the following characterization.

\begin{theorem}
	\label{thm:free_natural_filter}
	Let $\cL$ be a family of PLDs, $B$ a PLD, and $k > 0$. For any $L_1, \dots, L_{i} \in \cL$, let
	\begin{align*}
		V_{k - i}(L_1, \dots, L_{i}) := \sup\{L_{i + 1} \oplus V_{k - (i + 1)}(L_1, \dots, L_{i + 1}) : L_{i + 1} \in \cL, L_1 \oplus \dots \oplus L_{i + 1} \preceq B\}
	\end{align*}
	where $V_0(L_1, \dots, L_k) := \Iddist$. The natural filter for $k$ queries in $\cL$ with budget $B$ is free iff $V_k(\lambda) \preceq B$.
\end{theorem}

Informally, $V_{k - i}(L_1, \dots, L_{i})$ is the maximal remaining privacy cost of an adversary that has already played moves $L_1, \dots, L_{i}$. Naturally, when the adversary has already selected all of its moves, the remaining privacy cost is $\Iddist$, i.e. nothing.

It is also of interest to understand when a given family of queries $\cL$ simultaneously admits free natural filters against an arbitrary budget $B$. It turns out that this is the case exactly when the supremum commutes with the convolution operator.

\begin{theorem}
	\label{thm:universal_free_natural_filter}
	Fix a family of PLDs $\cL$ and a query capacity $k$. The natural filter \Cref{alg:natural_filter} is free for choices of budgets $B$ if and only if
	\begin{align*}
		\forall L \in \cL \ \textrm{and} \ \cL' \subseteq \cL^{\infty} \quad
			L \oplus \sup{\cL'} = \sup\{L \oplus L' : L' \in \cL'\}.
	\end{align*}
\end{theorem}

Our key result is that that this commutativity condition can essentially only be satisfied when $\cL^\infty$ is well-ordered.

\begin{theorem}
	\label{thm:well_ordering}
	Fix a family of non-degenerate PLDs $\cL$, i.e. every $L \in \cL$ is supported at at least two points in $\bR$. The natural filter \Cref{alg:natural_filter} is free for every budget $B$ and every query capacity $k$ if and only if $\cL^\infty$ is well-ordered with respect to $\preceq$.
\end{theorem}

It is straightforward to show that families of queries that are well-ordered under composition satisfy the commutativity condition of \Cref{thm:universal_free_natural_filter}. The reverse direction is significantly more challenging. The key technical idea is that any PLD whose support can be shifted to hit gaps between other PLDs causes those gaps to align when composed. By analyzing the topology of a pair of privacy profiles that are not well-ordered, one can show that this support shifting condition is satisfied.

	\section{Notable Implications}
\label{sec:implications-special-case}

\subsection{Natural filters are NOT Free in General}

% \Cref{thm:universal_free_natural_filter,thm:well_ordering} give a necessary and sufficient condition for free PLD filters.
% We now switch our view to $f$-DP using \Cref{prop:f-hs-equivalence,prop:sup-conv}, for which free filters correspond to
% \Cref{alg:fdp_natural_filter} in Appendix, to illustrate these results and their implications.
% The condition in \Cref{thm:universal_free_natural_filter}, which is necessary and sufficient for free privacy filters, translates as:
% \begin{equation}\label{eq:thm-free-filter-fdp}
%     \forall f \in \cF, \ \text{and} \ \cF' \subseteq \cF^\infty, f \otimes \conv(\cF') = \conv(f \otimes \cF'), 
% \end{equation}
% where $\otimes$ is $f$-DP composition, $f \otimes \cF' = \{f \otimes f': f' \in \cF' \}$, and $\conv$ is the lower convex envelope of a set of tradeoff curves.

A first implication \Cref{thm:universal_free_natural_filter,thm:well_ordering} is that there are no free $f$-DP filters in general:
\begin{corollary}[$f$-DP filters are NOT free]\label{corollary:no-free-filters}
    Consider $\cT$ the family of all tradeoff curves, and $\cF = \{f_{\epsilon, \delta}, \ \epsilon \geq 0, \ \delta \in [0,1] \}$ the set of tradeoff curves for all $(\epsilon, \delta)$-DP mechanisms.
    Neither $\cT$ nor $\cF$ have free $f$-DP filters.
    % Precisely, \Cref{alg:fdp_natural_filter} is not $f$-DP.
\end{corollary}
% \begin{proof}
%     $\cF \subset \cT$, and $\cF$ has crossing tradeoff functions, which are thus not well ordered. By \Cref{thm:well_ordering}, \Cref{alg:natural_filter} is not PLD for budget $B$, and so by \Cref{prop:f-hs-equivalence,prop:sup-conv} \Cref{alg:natural_filter} is not $f$-DP.
% \end{proof}

% \Cref{fig:counter-example} directly illustrates a counter-example to \Cref{thm:universal_free_natural_filter}  (\Cref{eq:thm-free-filter-fdp}) visually. We show that there exists tradeoff curves in $\cF$ (approximate DP mechanisms) such that $\alpha \ \textrm{s.t.} \ g_1 \otimes \conv(g_1, g_2)(\alpha) < \conv(g_1 \otimes g_1, g_1 \otimes g_2)(\alpha)$. That is, the exists $f=g1$ and $\cH = \{g_1, g_2\}$ such that $f \otimes \conv(\cH) \neq \conv(f \otimes \cH)$, violating (\Cref{eq:thm-free-filter-fdp}). By \Cref{prop:f-hs-equivalence,prop:sup-conv}, there exists $L$ and $\cL'$ such that $L \oplus \sup{\cL'} \neq \sup\{L \oplus L' : L' \in \cL'\}$, violating \Cref{thm:universal_free_natural_filter}.
% %
% In fact, \Cref{fig:counter-example} illustrates the two cases to account for, to directly prove a special case of \Cref{thm:well_ordering} for $(\epsilon, \delta)$-DP:

% \begin{proposition}[$(\epsilon, \delta)$-DP curves that cross always violate \Cref{thm:universal_free_natural_filter} (\Cref{eq:thm-free-filter-fdp})]\label{prop:no-commutativity-for-eps-delta}
%     Consider $f \in \cT$, and any set of tradeoff curves $\cF$ such that $f_{\epsilon_1, \delta_1}$ and $f_{\epsilon_2, \delta_2}$ cross, and $f_{\epsilon_1, \delta_1}, f_{\epsilon_2, \delta_2} \in \cF$.
%     $\cF$ does not have free $f$-DP filters.
%     % Precisely, \Cref{alg:fdp_natural_filter} is not $f$-DP.
% \end{proposition}
% \begin{proof}
%     Direct corollary of \Cref{thm:well_ordering}, self standing direct proof in \Cref{sec:appendix:well-ordered-eps-delt}.
% \end{proof}

% \begin{figure}[htbp]
%   \centering
%   \begin{subfigure}[t]{0.48\textwidth}
%     \centering
%     \includegraphics[width=\linewidth]{plots/counter-example.png}
%     \caption{Case 1: $g_1 \otimes g_1$ and $g_1 \otimes g_2$ cross.}
%     \label{subfig:counter-example-crossing}
%   \end{subfigure}\hfill
%   \begin{subfigure}[t]{0.48\textwidth}
%     \centering
%     \includegraphics[width=\linewidth]{plots/composed-curves-dont-cross.png}
%     \caption{Case 2: $g_1 \otimes g_1 \geq g_1 \otimes g_2$}
%     \label{subfig:counter-example-not-crossing}
%   \end{subfigure}
%   \caption{Counter examples to $\Cref{thm:universal_free_natural_filter}$ (\Cref{eq:thm-free-filter-fdp}), showing that $f$-DP filters are NOT free in general (for the set of all tradeoff curves $\cT$), or when choosing adaptively among the set of $(\epsilon, \delta)$-DP mechanisms. This is true even if $f\in\cF$.}
%   \label{fig:counter-example}
% \end{figure}

% Another interesting case is that of pure DP, or $\epsilon$-DP, mechanisms. Indeed, this family is well ordered ($\epsilon_1 > \epsilon_2 \geq 0 \Rightarrow f_{\epsilon_1} \preceq f_{\epsilon_2}$, illustrated on \Cref{subfig:pure-dp-twosteps}), and known to admit privacy filters under basic composition \cite{RRUV16}. However, this family of DP mechanisms is not closed under $f$-DP or PLD composition, and its self composition is not well ordered anymore, as illustrated on \Cref{subfig:pure-dp-counter-example}. As we can see, this is sufficient to violate \Cref{thm:universal_free_natural_filter,thm:well_ordering}, and hence rule out free $f$-DP/PLD filters for the family of $\epsilon$-DP under $f$-DP/PLD composition.

% \begin{figure}[htbp]
%   \centering
%   \begin{subfigure}[t]{0.48\textwidth}
%     \centering
%     \includegraphics[width=\linewidth]{plots/pure-dp-twosteps.png}
%     \caption{$\epsilon$-DP admits two-step filters (example)}
%     \label{subfig:pure-dp-twosteps}
%   \end{subfigure}\hfill
%   \begin{subfigure}[t]{0.48\textwidth}
%     \centering
%     \includegraphics[width=\linewidth]{plots/pure-dp-no-filter.png}
%     \caption{$\epsilon$-DP general filter counter-example}
%     \label{subfig:pure-dp-counter-example}
%   \end{subfigure}
%   \caption{$\epsilon$-DP mechanisms are well ordered, but not closed under composition.
%   If $\cF = \{ f_\epsilon: \ \epsilon \geq 0 \}$, then $\cF^2 \subset \cF^\infty$ is not well ordered (\Cref{subfig:pure-dp-twosteps}), and by \Cref{thm:well_ordering} pure DP mechanisms do not admit an $f$-DP/PLD filter (\Cref{subfig:pure-dp-counter-example}).}
%   \label{fig:counter-example-pure-dp}
%   % Plots here: https://colab.research.google.com/drive/15fvCEe53jpBkm-hCB9fbfd-jJX9zP2KM?usp=sharing
% \end{figure}

% We have shown that $f$-DP/PLD filters are not free in general (when $\cF$ is the set of all tradeoff curves), or for the set of $(\epsilon, \delta)$-DP mechanisms.
% These results are surprising, as all previous filter results were that filters are free (RDP/zCDP \cite{feldman2021individual,lecuyer2021practical}, GDP \cite{koskelaindividual,smith2022fullyadaptivecompositiongaussian}) or asymptotically free ($(\epsilon,\delta)$-DP \cite{rogers2016privacy, whitehouse2023fully}).
% One might hope that restricting $f$, the bound on privacy loss used as input to the $f$-DP filter, would be sufficient to ensure free filters. Two natural candidates come to mind. First, restricting $f \in \cF$, a practical condition of choosing privacy bounds in the set of tradeoff curves used in the filter. Second, we could restrict $f$ to a GDP curve $G_\mu$ (the tradeoff curve of a Gaussian mechanism), or an approximate GDP curve $f_{0, \delta} \otimes G_\mu$ to support approximate DP $f_{\epsilon, \delta}$ tradeoff curves in $\cF$. These are good candidates, as $\{G_\mu : \mu \in \bR^+\}$ is the only known set of tradeoff curves to have free $f$-DP filters \cite{koskelaindividual,smith2022fullyadaptivecompositiongaussian}.

% Unfortunately, neither restriction is sufficient:

\begin{corollary}\label{cor:no-free-filter-for-f-in-F}
    Consider the set of $(\epsilon, \delta)$-DP tradeoff curves $\cF = \{f_{\epsilon, \delta} : \epsilon \geq 0, \ \delta \in [0,1]\}$. $\cF$ does not have free $f$-DP filters even for $f \in \cF$.
\end{corollary}

% \begin{proof}
%     Consider two tradeoff curves that cross $g_1 = f_{\epsilon_1, \delta_1} \in \cF$ and $g_2 = f_{\epsilon_2, \delta_2} \cF$, that fall in \emph{case 1} in the proof of \Cref{prop:no-commutativity-for-eps-delta} (this is possible, but the proof is identical for \emph{case 2}, swapping the proper slope).

%     Pick $f = f_{\epsilon, \delta}$ with $\epsilon = \log\big(e^{2\epsilon_1} - (1+e^{\epsilon_1})\frac{\delta_2 - \delta_1}{\alpha_1^{\star}}\big)$ (that is, $e^{-\epsilon}$ is the first slope of $\conv(g_1 \otimes g_1, g_1 \otimes g_2)$) and $\delta = \delta_1+\delta_2 - \delta_1\delta_2$ (that is, the $\delta$ of $\conv(g_1 \otimes g_1, g_1 \otimes g_2)$).

%     Consider an adversary playing \Cref{alg:fdp_natural_filter} with a $g_1$ mechanism first, and then an adaptive choice between $g_1, g_2$. By construction, $\conv(g_1 \otimes g_1, g_1 \otimes g_2) \geq f$, so $g_1 \otimes g_1 \geq f$ and $g_1 \otimes g_2 \geq f$: \Cref{alg:fdp_natural_filter} always accepts the query.
    
%     From the proof of \Cref{thm:universal_free_natural_filter} we know that a two-steps filter can play $g_1 \otimes \conv(g_1, g_2)$.
%     By construction, we have that $f(\alpha) = \conv(g_1 \otimes g_1, g_1 \otimes g_2)(\alpha)$ on $\alpha \in (0, \alpha_1^{\star2}]$ 
%     From \Cref{subfig:counter-example-crossing} (and the proof of \Cref{prop:no-commutativity-for-eps-delta}) we know that on $\alpha \in (0, \alpha_1^{\star2}]$, we have $g_1 \otimes \conv(g_1, g_2)(\alpha) < \conv(g_1 \otimes g_1, g_1 \otimes g_2)(\alpha) = f(\alpha)$. \Cref{alg:fdp_natural_filter} is thus not $f$-DP, concluding the proof.
% \end{proof}

% \begin{figure}[htbp]
%   \centering
%   \begin{subfigure}[t]{0.48\textwidth}
%     \centering
%     \includegraphics[width=\linewidth]{plots/GDP-f-counter-example.png}
%     \caption{GDP $G_\mu$ filter bound}
%     \label{subfig:counter-example-pure-gdp}
%   \end{subfigure}\hfill
%   \begin{subfigure}[t]{0.48\textwidth}
%     \centering
%     \includegraphics[width=\linewidth]{plots/approxGDP-f-counter-example.png}
%     \caption{Approximate GDP $f_{0, \delta} \otimes G_\mu$ filter bound}
%     \label{subfig:counter-example-approx-gdp}
%   \end{subfigure}
%   \caption{Counter example to free $f$-DP filters when \(f\) is GDP (a) or approximate\ GDP (b).}
%   \label{fig:counter-example-GDP}
%   % Plots here: https://colab.research.google.com/drive/15fvCEe53jpBkm-hCB9fbfd-jJX9zP2KM?usp=sharing
% \end{figure}

\begin{corollary}\label{cor:no-free-filter-for-f-gaussian}
    The set of all tradeoff curves $\cT$ does not have free $f$-DP filters even for $f \in \{ G_\mu : \mu \geq 0 \}$.
    
    The set of $(\epsilon, \delta)$-DP tradeoff curves $\cF = \{f_{\epsilon, \delta} : \epsilon \geq 0, \ \delta \in [0,1]\}$ does not have free $f$-DP filters even for $f \in \{ f_{0, \delta} \otimes G_\mu : \mu \geq 0, \delta \in [0, 1] \}$.
\end{corollary}

% \begin{proof}
%     Following the reasoning of \Cref{cor:no-free-filter-for-f-in-F}, we show that we can construct a set of two tradeoff curves $g_1, g_2 \in \cF$, and a bound $f$ in the correct family, such that $\conv(g_1 \otimes g_1, g_1 \otimes g_2) \geq f$ but there exists values of $\alpha$ s.t. $g_1 \otimes \conv(g_1, g_2)(\alpha) < f$. \Cref{subfig:counter-example-pure-gdp} shows such a construction for $f \in \{ G_\mu : \mu \geq 0 \}$ and \Cref{subfig:counter-example-approx-gdp} for $f \in \{ f_{0, \delta} \otimes G_\mu : \mu \geq 0, \delta \in [0, 1] \}$.
%     All trade-off curves are (compositions of) piecewise linear functions, or Gaussian trade-offs. They can thus be computed exactly in closed-form.
% \end{proof}

\subsection{Families of Mechanisms with Free Filters}

\Cref{thm:well_ordering}, combined with \Cref{prop:f-hs-equivalence}, show that a family of DP mechanisms that is closed under composition, and with well ordered tradeoff curves, admits free $f$-DP (and PLD by \Cref{prop:f-hs-equivalence}) filters.

\begin{corollary}\label{cor:gaussian-filter}
    The set of all Gaussian tradeoff curves $G =  \{ G_\mu : \mu \geq 0 \}$ has free filters.
    % That is \Cref{alg:fdp_natural_filter} is $f$-DP.
\end{corollary}

% \begin{proof}
%     For any $\mu_1, \mu_2 \geq 0$, $G_{\mu_1} \otimes G_{\mu_2} = G_{\sqrt{\mu_1^2 + \mu_2^2}} \in G$.
%     If $\mu_1 > \mu_2$, then $G_{\mu_1} \preceq G_{\mu_2}$.
%     Gaussian mechanisms are closed under composition and well ordered: by \Cref{thm:well_ordering}, they have free $f$-DP/PLD filters, as already shown in \cite{koskelaindividual,smith2022fullyadaptivecompositiongaussian}.
% \end{proof}

\begin{corollary}\label{cor:pure-delta-filter}
    The set of all $0, \delta$-DP tradeoff curves $\cH =  \{ f_{0, \delta} : \delta \in [0, 1] \}$ has free filters.
    % That is \Cref{alg:fdp_natural_filter} if $f$-DP.
\end{corollary}

% \begin{proof}
%     For any $\delta_1, \delta_2 \geq 0$, $f_{0,\delta_1} \otimes f_{0,\delta_2} = f_{0,\delta_1 + \delta_2 - \delta_1\delta_2} \in \cH$.
%     If $\delta_1 > \delta_2$, then $f_{0,\delta_1} \preceq f_{0,\delta_2}$.
%     Pure $\delta$ mechanisms are closed under composition and well ordered: by \Cref{thm:well_ordering}, they have free $f$-DP/PLD filters.
% \end{proof}

% The family of subsampled Gaussian mechanisms with monotonously decreasing subsampling rates (that is, subsampling rate $p(\mu)$ is monotonously decreasing in $\mu$).
% As special cases, subsampled Gaussian mechanisms with fixed subsampling rate, and the subsampled Gaussian mechanisms with fixed $\mu$ (and arbitrary $p$) both have free $f$-DP filters.
% For instance, when running DP-SGD with $f$-DP accounting \ml{cite}, one can adaptively choose $\mu$ OR $p$ OR both according to a monotonously decreasing family within a lower-bound $f$, applying regular composition results or empirical computations of the tradeoff curve \ml{cite}.

% The family of subsampled Gaussian mechanisms with monotonously decreasing $\mu$ as a function of the subsampling rate (that is, $\mu(p)$ is monotonously decreasing in $p$). As a special case, a set of mechanisms that apply Gaussian noise with fixed variance, but have variable subsampling rate $p$, are 

% The family of $\big(\epsilon, \delta(\epsilon)\big)$-DP mechanisms where $\delta(\epsilon)$ is a monotonously increasing function of $\epsilon$ has free $f$-DP filters. In particular, the family $\epsilon$-DP mechanisms, and of $(\epsilon, \delta)$-DP mechanisms with fixed $\delta$, both have free $f$-DP filters. This is interesting, as it means that we can apply optimal (strong) composition in a privacy filter for $\epsilon$-DP (or any $(\epsilon, \delta)$-DP mechanisms with fixed $\delta$) (and hence run a privacy filter with empirical composition \ml{cite} or any other composition result), but not the family of all $(\epsilon, \delta)$-DP mechanisms.

	% \section{Lower and Upper Bounds for the Natural PLD Filter}

We consider the ``natural'' $(\eps, \delta)$-PLD filter where we simply compose the dominating pairs of the mechanisms and check that the composed dominating pairs satisfies $(\eps, \delta)$-DP. The formal description of the filter is given below in \Cref{alg:pld-privacy-filter}.

\begin{algorithm}[h!]
        \caption{$(\eps, \delta)$-PLD Filter}
        \label{alg:pld-privacy-filter}
        \begin{algorithmic}
            \REQUIRE Privacy budgets $\eps, \delta$, adversary (or analyst) $\adv$, number of steps $k$, Input Dataset $X$
            \STATE $t \gets 0$
            \WHILE{$\adv$ continues and $t < k$}
            \STATE $t \leftarrow t+1$
            \STATE $\adv$ gives mechanism $\mech_t$ with worst-case dominating pair $(P_t, Q_t)$  \algcomment{can depend on previous results $m_{<t}$}
            \IF{$H_{e^{\eps}}(P_1 \times \cdots \times P_t \| Q_1 \times \cdots \times Q_t) \leq \delta$}
            \STATE $r=\top$
            \ELSE
            \STATE $r=\bot$
            \STATE $P_t, Q_t \gets \Iddist$, $\mech_t \leftarrow \nullmech$
            \ENDIF
            \STATE $\adv$ receives $v_t \triangleq (r, m_t), \ m_t \sim \mech_t(X)$
            \ENDWHILE
            \RETURN $(v_1, \ldots, v_{t})$
        \end{algorithmic}
    \end{algorithm}

Our main result is that, while the above natural $(\eps, \delta)$-PLD filter can fail, it cannot fail \emph{too badly}; in the sense that we still get a DP guarantee with a blow-up in parameters that is only polylogarithmic in $k, 1/\delta$, as stated below.

\begin{theorem} \label{thm:main-ub}
The $(\eps,\delta)$-PLD filter (\Cref{alg:pld-privacy-filter}) satisfies $\Paren{\kappa \cdot (\eps + 1), \kappa \cdot \delta}$-DP where $$\kappa = O\Paren{\Paren{\log k + \log\Paren{1 + \frac{1}{\eps}}} \cdot \log(1/\delta)}.$$ 
\end{theorem}

When $\eps = \Omega(1)$, the above guarantee is simply $(\kappa' \cdot \eps, \kappa' \cdot \delta)$-DP where $\kappa' = O(\log k \cdot \log(1/\delta))$. 

\begin{remark}
We tradeoff the blow-up factors of the two parameters $\eps, \delta$. For example, by changing $\eps_i$ in \Cref{subsec:disc} to $\frac{2\eps}{2^{n - 1 - i}}$. This allows us to have a guarantee of $\left(\kappa \cdot \eps, \kappa \cdot \frac{\delta}{\eps} \right)$-DP for $\kappa = O(\log k \cdot \log(1/\delta))$.
\end{remark}

\begin{remark}
In our current proof, we do not try to optimize the parameter $\kappa$. In particular, it seems plausible that our techniques can achieve $\kappa = \tilde{O}(\log k + \log(1/\delta))$ with a more refined argument. Nevertheless, we are not aware of how to remove the dependency of $k$ or $\delta$ in $\kappa$, and this remains an interesting open question.
\end{remark}

\begin{remark}
The only property we need for worst-case dominating pair is that it is symmetric (\Cref{lem:worst-case-symmetric}) in order to apply \Cref{thm:optimistic,thm:pessimistic} below. Thus, our proof still works as long as $(P_t, Q_t)$ is any symmetric (\emph{not necessarily worst-case}) dominating pair.
\end{remark}

We have also a lower bound that nearly matches the upper bound in \Cref{thm:main-ub}.

\begin{theorem} \label{thm:natural-filter-lower}
For any $k > 1$, any sufficiently small $\delta > 0$ (depending on $k$) and any sufficiently large $\eps > 0$ (depending on $k$ but not in $\delta$), it is possible for the natural PLD filter to allow $$\E[D_{e^\eps}(X_1 + \cdots + X_k)] \geq \Omega\left(\frac{\ln(1/\delta)}{k}\right)^{k-1} \cdot \delta.$$
\end{theorem}


	\bibliographystyle{plain}
	\bibliography{bibliography}

	% \appendix
	% \section{Prelims}

For shorthand, we denote by $\pld(M) := \pld(M(D_1) \parallel M(D_2))$ the PLD of a mechanism $M$. Note that a distribution can be easily recognized as a PLD as follows.

\begin{proposition}
    A distribution $L$ on $\bR \cup \{\infty\}$ is the PLD of some $(P, Q$) if and only if it satisfies $\E_{Z \sim L}[e^{-Z}] \leq 1$. In this case, one such pair is $(L, L')$ where $L'$ is the Esscher tilt of $L$, namely $dL'(z) := e^{-z} dL(z)$ for $z \in \bR$ and $L'(\{-\infty\}) := 1 - \E_{Z \sim L}[e^{-Z}]$.
\end{proposition}

PLDs also inherit the Blackwell order: For PLDs $L$ and $L'$, we say that $L \preceq L'$ if there exist pairs $(P, Q)$ and $(P', Q')$ such that $L := \pld(P \parallel Q)$, $L' = \pld(P' \parallel Q')$, and $(P, Q) \preceq (P', Q')$. It is straightforward to show that this order is well-defined regardless of the underlying representations $(P, Q)$ and $(P', Q')$.

PLDs also form a commutative monoid under convolution that is monotone in the Blackwell order.

\begin{proposition}
    \label{prop:pld_convolution_properties}
    For any PLDs $L_1, L_2, L_3, L'_1, L'_2$,
    \begin{enumerate}
        \item The convolution $L_1 \oplus L_2$ is also a PLD;
        \item The identically zero distribution $\Iddist$ is a PLD such that $L_1 \oplus \Iddist = \Iddist \oplus L_1 = L_1$;
        \item $(L_1 \oplus L_2) \oplus L_3 = L_1 \oplus (L_2 \oplus L_3)$;
        \item $L_1 \oplus L_2 = L_2 \oplus L_1$; and
        \item If $L_1 \preceq L'_1$ and $L_2 \preceq L'_2$, then $L_1 \oplus L_2 \preceq L'_1 \oplus L'_2$.
    \end{enumerate}
\end{proposition}

A highly prized property of PLDs is that the privacy characteristics of an adaptive mechanism corresponds to convolution of PLDs, provided that each component fixes a privacy budget in advance. This is important because the convolution of a long sequence of distributions can be computed very efficiently by applying the fast Fourier transform \cite{koskela2020computing}. Note that we will require more than this to understand fully adaptive composition where the privacy bound of each component is itself adaptive.

\begin{proposition}
    \label{prop:composition_convolution}
    Let $M_1 : \cD \to \cY_1$ and $M_2 : \cD \times \cY_1 \to \cY_2$ be adaptive mechanisms such that $\pld(M_1) \preceq L_1$ and $\pld(M_2(\cdot; y_1)) \preceq L_2$ for every $y_1 \in \cY_1$. Then $\pld(M_1 \otimes M_2) \preceq L_1 \oplus L_2$. Moreover, if $\pld(M_1) \preceq \pld(M'_1)$ and $\pld(M_2(\cdot; y_1)) \preceq \pld(M'_2(\cdot; y_1))$ for every $y_1 \in \cY_1$, then $\pld(M_1 \otimes M_2) \preceq \pld(M'_1 \otimes M'_2)$.
\end{proposition}

Closely related to the PLD is the hockey-stick curve: a convex reparameterization of the privacy profile of a mechanism generalizing the classical notion of $(\eps, \delta)$-DP.

\begin{definition}
    For a pair of distributions $(P, Q)$ over $\Omega$, their hockey-stick divergence at order $x \in \bR^\times := (0, \infty)$ is
    \begin{align*}
        H_x(P \parallel Q) := \sup_{E \subseteq \Omega} P(E) - xQ(E).
    \end{align*}
\end{definition}

Again, for convenience, we write $H_x(M) := H_x(M(D_1) \parallel M(D_2))$ and sometimes $h_M(x) := H_x(M)$ for the hockey-stick curve of a mechanism $M$. We will rely on hockey-stick curves in order to reason about adversarial behaviour of adaptive mechanisms. The hockey-stick curve is very closely related to the classical $(\eps, \delta)$ form of differential privacy. In particular, by construction of the hockey-stick divergence a mechanism $M$ satisfies $(\eps, \delta)$-DP exactly when $H_{e^\eps}(M) \leq \delta$. In general, hockey-stick curves are characterized by a few simple conditions.

\begin{proposition}[\cite{ZhuDW22} Lemma 9]
    \label{prop:hs_characterization}
    A curve $h : \bR^\times \to [0, 1]$ is a hockey-stick curve of some pair $(P, Q)$ if and only if $h$ is convex and decreasing such that $\lim_{x \to 0}h(x) = 1$ and $h(x) \geq 1 - x$ for all $x \in \bR^\times$.
\end{proposition}

Hockey-stick curves are also closely related to PLDs and can be fully recovered from the PLD via the following formula.

\begin{proposition}
    \label{prop:pld_to_hs}
    For any mechanism $M$ and $x \in \bR^\times$,
    \begin{align*}
        h_M(x) = \E_{Z \sim \pld(M)}[(1 - xe^{-Z})_+]
    \end{align*}
    where $(t)_+ := \max\{0, t\}$.
\end{proposition}

\begin{proof}
    For any pair $(P, Q)$ defined on $\Omega$ with likelihood ratio $\ell$, we have that
    \begin{align*}
        H_x(P \parallel Q)
            & = \sup_{E \subseteq \Omega} P(E) - xQ(E) \\
            & = \sup_{E \subseteq \Omega} \int_E \frac{d(P - xQ)}{dP}(\omega) \, dP(\omega) \\
            & = \sup_{E \subseteq \Omega} \int_E 1 - x/\frac{dP}{dQ}(\omega) \, dP(\omega) \\
            & = \int_\Omega \left(1 - x/\frac{dP}{dQ}(\omega)\right)_+ \, dP(\omega) \\
            & = \int_\Omega (1 - xe^{-\log(\ell(\omega))})_+ \, dP(\omega) \\
            & = \E_{Z \sim \pld(P \parallel Q)}[(1 - xe^{-Z})_+].
    \end{align*}
\end{proof}

For convenience, we will sometimes write $h_L$ to denote the unique hockey-stick curve associated to a PLD $L$, namely $h_L(x) := \E_{Z \sim L}[(1 - xe^{-Z})_+]$. Moreover, like PLDs, hockey-stick curves also capture adaptive composition in a natural way.
% Doing so computationally involves expensive numerical integration and is not generally practical.
% \todo{explain that it's useful anyway 'cause the formula is more flexible than for plds}

\begin{proposition}
    \label{prop:hockey_stick_composition}
    Let $M_1 : \cD \to \cY_1$ and $M_2 : \cD \times \cY_1 \to \cY_2$ be adaptive mechanisms and let $\ell^1 := \frac{dM_1(D_1)}{dM_1(D_2)}$ denote the likelihood ratio for $M_1$. Then
    \begin{align*}
        h_{M_1 \otimes M_2}(x) = \E_{Y_1 \sim M_1(D_1)}[h_{M_2(\cdot; Y_1)}(x/\ell^1(Y_1))].
    \end{align*}
\end{proposition}

\begin{proof}
    Let $\ell$ denote the likelihood ratio for $M_1 \otimes M_2$ and let $\ell^2_{y_1}$ denote the likelihood ratio for $M_2(\cdot; y_1)$. By Bayes' rule, we have $\ell(y_1, y_2) = \ell^1(y_1) \cdot \ell^2_{y_1}(y_2)$, so we have
    \begin{align*}
        \pld(M_1 \otimes M_2) \equiv \log(\ell^1(Y_1)) + \log(\ell^2_{Y_1}(Y_2)), Y_1 \sim M_1(D_1), Y_2 \sim M_2(D_1; Y_1).
    \end{align*}
    By \Cref{prop:pld_to_hs} and the law of total expectation, it follows that
    \begin{align*}
        h_{M_1 \otimes M_2}(x)
            & = \E_{Z \sim \pld(M_1 \otimes M_2)}[
                        (1 - xe^{-Z})_+
                    ] \\
            & = \E_{Y_1 \sim M_1(D_1)}[\E_{Y_2 \sim M_2(D_1; Y_1)}[
                        (1 - xe^{-(\log(\ell^1(Y_1)) + \log(\ell^2_{Y_1}(Y_2)))})_+
                    ]] \\
            & = \E_{Y_1 \sim M_1(D_1)}[\E_{Y_2 \sim M_2(D_1; Y_1)}[
                        (1 - x/\ell^1(Y_1) \cdot e^{-\log(\ell^2_{Y_1}(Y_2))})_+
                    ]] \\
            & = \E_{Y_1 \sim M_1(D_1)}[\E_{Z_2 \sim \pld(M_2(\cdot; Y_1)}[
                        (1 - x/\ell^1(Y_1) \cdot e^{-Z_2})_+
                    ]] \\
            & = \E_{Y_1 \sim M_1(D_1)}[h_{M_2(\cdot; Y_1)}(x/\ell^1(Y_1))].
    \end{align*}
\end{proof}

Our last characterization of privacy is given the Type I/Type II error tradeoff curve, also known as the $f$-DP framework of privacy \cite{DongRS19}.

\begin{definition}
    Let $(P, Q)$ be any distributions on $\Omega$. For any hypothesis test $\phi : \Omega \to \{P, Q\}$ for distinguishing $P$ and $Q$, we call $\alpha_\phi := \bP_{\omega \sim P}[\phi(\omega) = Q]$ its Type I error and $\beta_\phi := \bP_{\omega \sim Q}[\phi(\omega) = P]$ its Type II error. The tradeoff curve for $(P, Q)$ is defined as
    \begin{align*}
        T_\alpha(P \parallel Q) := \inf_{\phi} \{\beta_\phi : \alpha_\phi \leq \alpha\}.
    \end{align*}
\end{definition}

For convenience, we sometimes write the tradeoff curve of a mechanism $M$ with respect to the fixed dataset pair $(D_1, D_2)$ as $T_\alpha(M) := T_\alpha(M(D_1) \parallel M(D_2))$ or sometimes as $\tau_M(\alpha) := T_\alpha(M)$ where appropriate. In general, we can characterize valid tradeoff curves as follows.

\begin{proposition}[\cite{DongRS19} Proposition 2.2]
    \label{prop:tradeoff_characterization}
    A curve $\tau : [0, 1] \to [0, 1]$ is a tradeoff curve for some testing problem $(P, Q)$ exactly when $\tau$ is continuous, convex, decreasing, and $\tau(\alpha) \leq 1 - \alpha$ for all $\alpha \in [0, 1]$.
\end{proposition}

Like hockey-stick curves, it is enough to understand the PLD of a testing problem in order to obtain its tradeoff curve. The following formula can be derived by the well-known Neyman--Pearson lemma.

\begin{proposition}
    \label{prop:pld_to_tradeoff}
    Let $(P, Q)$ be a pair of distributions with tradeoff curve $\tau(\alpha) := T_\alpha(P \parallel Q)$ and let $F$ denote the CDF of $\pld(P \parallel Q)$. Then
    \begin{align*}
        \tau(\alpha) = \mathbb{E}_{Z \sim \pld(P \parallel Q)}[1(F(Z) > \alpha) \cdot e^{-Z}]
    \end{align*}
\end{proposition}

It is common to represent a pair of distributions by either their hockey-stick curve or their tradeoff function. We show that there is a natural link between these representations via inversion and convex conjugacy. For a tradeoff function $\tau$ of a testing problem $(P, Q)$, we write $\tau^{-1}(\beta) := \inf\{\alpha \in [0, 1] : \tau(\alpha) \leq \beta\}$ for its inverse, namely the tradeoff curve for the testing problem $(Q, P)$. For a convex function $g$, we denote its convex conjugate by $g^*(y) := \sup_{x \in \mathbb{R}} xy - g(x)$. Note that we can extend tradeoff functions to the real-line by setting them to $\infty$ outside their support $[0, 1]$. Note that a special case of the following formula appears in \cite{DongRS19} (Prop 2.12) for symmetric tradeoff functions but this generalization is new to the best of our knowledge.

\begin{proposition}
    \label{prop:tradeoff_to_hs}
    Let $(P, Q)$ be a pair of distributions with tradeoff function $\tau(\alpha) := T_\alpha(P \parallel Q)$ and hockey-stick curve $h(x) := H_x(P \parallel Q)$. Then
    \begin{align*}
        h(x) = 1 + (\tau^{-1})^*(-x)
    \end{align*}
\end{proposition}

% \matt{note: for now the proof skips lots of little details, mostly relating to edge cases involving point masses and how it affects $\tau^{-1}$, $F_L^{-1}$ etc.}

\begin{proof}
    First, let $L$ denote the PLD of $(P, Q)$ and let $F$ be its CDF. By \Cref{prop:pld_to_tradeoff},
    \begin{align*}
        \tau(\alpha) = \mathbb{E}_{Z \sim L}[1(Z > F^{-1}(\alpha)) \cdot e^{-Z}].
    \end{align*}
    Therefore, by \Cref{prop:pld_to_hs}
    \begin{align*}
        h(x)
            & = \mathbb{E}_{Z \sim L}[(1 - x e^{-Z})_+] \\
            & = \mathbb{E}_{Z \sim L}[1(Z > \log(x)) \cdot \underbrace{(1 - x e^{-Z})}_{> 0 \iff Z > \log(x)}] \\
            & = \sup_{\varepsilon \in \mathbb{R}}\mathbb{E}_{Z \sim L}[1(Z > \varepsilon) \cdot (1 - x e^{-Z})] \\
            & = \sup_{\alpha \in [0, 1]}\mathbb{E}_{Z \sim L}[1(Z > F^{-1}(\alpha)) \cdot (1 - x e^{-Z})] \\
            & = \sup_{\alpha \in [0, 1]} 1 - \alpha - x\tau(\alpha) \\
            & = \sup_{\beta \in [0, 1]} 1 - \tau^{-1}(\beta) - x \beta \tag{$\alpha = \tau^{-1}(\beta)$} \\
            & = 1 + (\tau^{-1})^*(-x).
    \end{align*}
\end{proof}

It follows as a natural consequence that the Blackwell order on pairs, the ordering induced by the tradeoff function, as well as the hockey-stick induced ordering are all equivalent.

\begin{proposition}
    \label{prop:f-hs-equivalence}
    Let $(P, Q)$ and $(P', Q')$ be pairs of distributions with tradeoff functions $\tau$ and $\tau'$ respectively as well as hockey-stick curves $h$ and $h'$ respectively. The following are equivalent
    \begin{enumerate}[(i)]
        \item $(P, Q) \preceq (P', Q')$
        \item $\tau \succeq \tau'$
        \item $h \preceq h'$
    \end{enumerate}
\end{proposition}

The equivalence of (i) and (ii) is given by a celebrated theorem of Blackwell \cite[Theorem 10]{Bla51}. As for the equivalence of (ii) and (iii), this now follows by applying Fenchel–Moreau duality to \Cref{prop:tradeoff_to_hs} and recalling that convex conjugation is an order-reversing operation.

As a consequence, it turns out that PLDs endowed with the Blackwell order possess the same remarkable completeness property that enables analysis in the real numbers.

\begin{proposition}\label{prop:sup-conv}
    For any non-empty family of PLDs $\mathcal{L}$ dominated by at least one PLD, there exists a unique PLD $\sup \mathcal{L}$ that dominates $\mathcal{L}$ but is dominated by every upper bound for $\mathcal{L}$. In this case,
    \begin{enumerate}[(i)]
        \item $h_{\sup \mathcal{L}}(x) = \sup\{h_L(x) : L \in \mathcal{L}\}$
        \item $\tau_{\sup \mathcal{L}} = \conv\{\tau_L : L \in \mathcal{L}\}$
    \end{enumerate}
    where $(\conv{\cF})(\alpha) := \sup\{f(\alpha) : f \text{ is convex}, f \preceq \cF\}$ denotes the lower convex envelope.
\end{proposition}

This result follows immediately from \Cref{prop:f-hs-equivalence} and from noticing that the properties characterizing hockey-stick curves (see \Cref{prop:hs_characterization}) are all closed under pointwise suprema. Part (ii) follows from noticing that all of the properties of tradeoff curves (see \Cref{prop:tradeoff_characterization}) are also closed under the lower convex envelope operation.

We rely on the supremum extensively in order to reason about natural privacy filters. In particular, the former characterization says that the supremum of PLDs is given by a pointwise supremum in the space of hockey-stick curves, which will be particularly useful for showing our main result. As far as we are aware, the supremum property of PLDs has not actually been documented in the literature.

	% \section{Proof of \Cref{lemma:restricted_adversary}}
\label{app:restricted_adversary_proof}

\begin{proof}
    The case where $k = 1$ is immediate, so we assume $k > 1$. We first show that
    \begin{align*}
        \pld(\Gamma) \preceq \sup\{L \oplus \pld(\Gamma_L) : L \in \Gamma(\lambda)\}.
    \end{align*}
    To that end, consider any $\Gamma$-adversary $M$. Decompose $M$ as the adaptive composition of $M_1(\cdot)$ with some $M'(\cdot; y_1)$ and let $L := \pld(M_1) \in \Gamma(\lambda)$. Then, clearly, for every fixed $y_1$, $M'(\cdot; y_1)$ is a $\Gamma_L$-adversary and, in particular, $\pld(M'(\cdot; y_1)) \preceq \pld(\Gamma_L)$. By \Cref{prop:composition_convolution}, we have that $\pld(M) \preceq L \otimes \pld(\Gamma_L)$ and the desired inequality follows by taking suprema.

    More surprising is the reverse inequality, which we prove constructively by passing to hockey-stick curves. Let $L \in \Gamma(\lambda)$, let $x \in \bR^\times$, and let $\gamma > 0$. Choose any mechanism $M_1$ that has PLD $L$ and let $\ell_1 := \frac{dM_1(D_1)}{dM_1(D_2)}$ denote the corresponding likelihood function. Moreover, for any $x \in \R^\times$ we can choose by \Cref{prop:sup-conv}, a $\Gamma_L$-adversary $M^x$ such that
    \begin{align*}
        H_x(M^x) \geq \sup_{M' \in \Adv(\Gamma_L)}H_x(M') - \gamma.
    \end{align*}
    Now consider the $\Gamma$-adversary $M := M_1 \otimes M_2$ where $M_2(D; y_1) := M^{x/\ell_1(y_1)}(D)$. By \Cref{prop:hockey_stick_composition}, we get
    \begin{align*}
        h_M(x)
            & = \E_{Y_1 \sim M_1(D_1)}[h_{M_{x/\ell_1(Y_1)}}(x/\ell_1(Y_1))] \\
            & \geq \E_{Y_1 \sim M_1(D_1)}[\sup_{M' \in \Adv(\Gamma_L)}h_{M'}(x/\ell_1(Y_1)) - \gamma] \\
            & = (h_{M_1} \otimes \sup_{M' \in \Adv(\Gamma_L)} h_{M'})(x) - \gamma.
    \end{align*}
    Again, taking suprema, we get that
    \begin{align*}
        \sup_{M \in \Adv(\Gamma)} h_M \succeq h_{M_1} \otimes \sup_{M' \in \Adv(\Gamma_L)} h_{M'},
    \end{align*}
    and thus $\pld(\Gamma) \succeq L \otimes \pld(\Gamma_L)$, which completes the proof since $L$ was arbitrary.
\end{proof}
	% \section{A Proof of \Cref{thm:well_ordering}}
\label{sec:well_ordering_proof}

% \matt{integrate into \Cref{sec:natural_filter_characterization}??}

In this section, we give a proof of \Cref{thm:well_ordering} that goes through the topology of hockey-stick curves.

As we have seen, \Cref{thm:well_ordering} can be proved by showing that, for any family $\mathcal{F}$ that is not well-ordered, we can find $g_1, g_2 \in \mathcal{F}^\infty$ for which
\begin{align*}
    g_1 \otimes \conv\{g_1, g_2\} \prec \conv\{g_1 \otimes g_1, g_1 \otimes g_2\}
\end{align*}
Equivalently, in the language of PLDs, we would like to show that if $L_1$ and $L_2$ are not well-ordered, then $L_1 \oplus \sup\{L_1, L_2\} \succ \sup\{L_1 \oplus L_1, L_1 \oplus L_2\}$.

Our first step is to show that gaps between PLDs can be amplified in some sense by composition. For a distribution $A$ in the reals, we denote by $\supp(A) := \{t \in \mathbb{R} : \forall \text{open } U \ni t, A(U) > 0\}$ the support of $A$, namely the points with locally positive probability mass. It turns out that any PLD whose support can be shifted to hit gaps between other PLDs causes the gaps to align when composed.

\begin{proposition}[PLD Gap Amplification]
    \label{prop:pld_gap_amplification}
    Let $A, B, B_1, B_2$ be PLDs so that $B_i \prec B$
    with $h_{B_i}(e^{\varepsilon_i + u}) < h_B(e^{\varepsilon_i + u})$ for some $\varepsilon_1, \varepsilon_2 \in \supp(A)$ and some $u \in \mathbb{R}$. Then
    \begin{align*}
        A \oplus B \succ \sup\{A \oplus B_1, A \oplus B_2\}.
    \end{align*}
\end{proposition}

\begin{proof}
    First, it is clear that $A \oplus B \succeq \sup\{A \oplus B_1, A \oplus B_2\}$, so we just need to find a single gap.

    To that end, by continuity, we can find a constant $\rho > 0$ s.t $\inf_{\varepsilon \in (\varepsilon_i + u \pm \rho)} h_B(e^\varepsilon) - h_{B_i}(e^\varepsilon) > 0$. Now, set $\varepsilon^\star := \varepsilon_1 + \varepsilon_2 + u$ and notice that
    \begin{align*}
        h_{A \oplus B}(e^{\varepsilon^\star}) - h_{A \oplus B_i}(e^{\varepsilon^\star})
            & = \mathbb{E}_{Z \sim A \oplus B}[(1 - e^{\varepsilon^\star - Z})_+] - \mathbb{E}_{Z \sim A \oplus B_i}[(1 - e^{\varepsilon^\star - Z})_+] \\
            & = \mathbb{E}_{Z_1 \sim A}[\mathbb{E}_{Z_2 \sim B}[(1 - e^{(\varepsilon^\star - Z_1) - Z_2})_+] - \mathbb{E}_{Z_2 \sim B_i}[(1 - e^{(\varepsilon^\star - Z_1) - Z_2})_+]] \\
            & = \mathbb{E}_{Z \sim A}[h_B(e^{\varepsilon^\star - Z}) - h_{B_i}(e^{\varepsilon^\star - Z})] \\
            & \geq \mathbb{P}_{Z \sim A}(Z \in (\varepsilon_{2 - i} \pm \rho)) \cdot \inf_{\varepsilon \in (\varepsilon_{2 - i} \pm \rho)} h_B(e^{\varepsilon_1 + \varepsilon_2 + u - \varepsilon}) - h_{B_i}(e^{\varepsilon_1 + \varepsilon_2 + u - \varepsilon}) \\
            & = \mathbb{P}_{Z \sim A}(Z \in (\varepsilon_{2 - i} \pm \rho)) \cdot \inf_{\varepsilon \in (\varepsilon_i + u \pm \rho)} h_B(e^\varepsilon) - h_{B_i}(e^\varepsilon) \\
            & > 0.
    \end{align*}
    In particular,
    $h_{\sup\{A \oplus B_1, A \oplus B_2\}}(e^{\varepsilon^\star}) = \max\{h_{A \oplus B_1}(e^{\varepsilon^\star}), h_{A \oplus B_2}(e^{\varepsilon^\star})\}
        < h_{A \oplus B}(e^{\varepsilon^\star})$.
    % \begin{align*}
    %     h_{\sup\{A \oplus B_1, A \oplus B_2\}}(e^{\varepsilon^\star}) = \max\{h_{A \oplus B_1}(e^{\varepsilon^\star}), h_{A \oplus B_2}(e^{\varepsilon^\star})\}
    %     < h_{A \oplus B}(e^{\varepsilon^\star}).
    % \end{align*}
\end{proof}

In order to apply gap amplification to PLDs that are not well-ordered i.e. $L_1, L_2 \prec \sup\{L_1, L_2\}$, we would like to reduce the problem to geometric and topological properties of convex curves over $\bR^\times := (0, \infty)$. To that end, we also notice that the support of a PLD is determined by the topology of its hockey-stick curve. This result follows immediately from \Cref{prop:pld_to_hs}

\begin{proposition}
    \label{prop:support_conversion}
    For a PLD $L$ with hockey-stick curve $h_L$, denote by $\supp(h_L) := \{x \in \bR^\times : \forall \text{open }U \ni x, h|_U \text{ is not affine}\}$ the points at which $h_L$ ``bends''. Then for any $e^\varepsilon \in \supp(h_L)$, we have $\varepsilon \in \supp(L)$.
\end{proposition}

% \begin{proof}
%     \todo{appeal to $h_L(x) = \mathbb{E}_{Z \sim L}[(1 - x e^{-Z})_+]$.}
% \end{proof}

Our problem is now purely geometric. Given decreasing, convex, unordered curves $h_1, h_2 : \bR^\times \to [0, 1]$ that meet in the limit $x \to 0$, we would like to show that one of them can be scaled along the x-axis so that its elbows hit a region where $h_1 < h_2$ and another region where $h_1 > h_2$.

We proceed by analysis of these regions' topology within the ambient space $\bR^\times$. Given $S \subseteq \bR^\times$, let $\partial S$ denote its boundary, i.e. those points whose neighbourhoods meet both $S$ and $S^C := \bR^\times \setminus S$, let $\mathrm{int}(S) := S \setminus \partial S$ denote its interior, and let $\overline{S} := S \cup \partial S$ denote its closure. Now, let
\begin{align*}
    X(h_1, h_2) := \{x \in \bR^\times : h_1(x) < h_2(x)\},
\end{align*}
let $X(h_2, h_1)$ be defined similarly, and let $E(h_1, h_2) := (X(h_1, h_2) \cup X(h_2, h_1))^C$ be the points where $h_1$ and $h_2$ agree. Note that, convex curves over $\bR^\times$ are also continuous, so $X(h_1, h_2)$ and $X(h_2, h_1)$ are both open and $E(h_1, h_2)$ is closed. Finally, let
\begin{align*}
    C(h_1, h_2) := \{x \in \bR^\times : (x r^{-1}, x) \subseteq X(h_1, h_2), (x, x r) \subseteq X(h_2, h_1) \text{ or vice-versa for some } r > 1\}
\end{align*}
denote the \textit{simple} crossing points of $h_1$ and $h_2$. We would like to show that either $\supp(h_1)$ or $\supp(h_2)$ ``links'' $X(h_1, h_2)$ with $X(h_2, h_1)$ in the following sense.

\begin{definition}
    For subsets $S, T_1, T_2$ of $\bR^\times$, we say that $S$ (multiplicatively) links $T_1$ and $T_2$ if there is $c > 0$ so that $c \cdot S \cap T_1 \neq \varnothing$ and $c \cdot S \cap T_2 \neq \varnothing$. Furthermore, for $S, T \subseteq \bR^\times$, we say that $S$ divides $T$, written $S \mid T$, if $S$ does not link $T$ and $T^C$.
\end{definition}

In some cases, it will be easier to directly link a set with the boundary of another set. The following lemma shows that this will suffice for our purposes.

\begin{lemma}
    \label{lemma:closure_linkage}
    Let $S, U_1, U_2 \subseteq \bR^\times$ with $U_1$ and $U_2$ open. The following are equivalent.
    \begin{enumerate}[(i)]
        \item $S$ links $U_1$ and $U_2$
        \item $S$ links $U_1$ and $\overline{U_2}$
        \item $S$ links $\overline{U_1}$ and $U_2$
    \end{enumerate}
\end{lemma}

\begin{proof}
    Clearly, (i) implies (ii) and (iii). Moreover, (ii) and (iii) are symmetric, so we just need to show (ii) $\implies$ (i). To that end, assume that there is $x_1, x_2 \in S$ and $c > 0$ such that $cx_1 \in U_1$ and $cx_2 \in \overline{U_2}$. Now, if $cx_2 \in U_2$, we are done, so assume $cx_2 \in \partial U_2$. That is, any arbitrarily small neighbourhood of $cx_2$ meets $U_2$, so, since $U_1 \ni cx_1$ is open, we can just choose $c' > 0$ sufficiently close to $c$ so that $c'x_1$ remains in $U_1$ and $c'x_2 \in U_2$.
\end{proof}

Critically, non-trivial closed and divisible subsets of $\bR^\times$ have a highly regular multiplicative structure enclosed byt their divisors.

\begin{proposition}
    \label{prop:dense_subgroup}
    Let $S \subseteq \bR^\times$ contain more than one point and let $\varnothing \subsetneq T \subsetneq \bR^\times$ be closed. Then, if $S \mid T$, there is $a > 1$ and $B \subseteq [1, a)$ such that $T = a^\bZ B = \{a^k b : k \in \bZ, b \in B\}$ and such that $a \leq s'/s$ for any $s < s' \in S$.
\end{proposition}

\begin{proof}
    We first claim that $(s'/s)^\bZ T \subseteq T$ for any $s, s' \in S$. Indeed, let $t \in T$ and $s, s' \in S$. Then $t/s \cdot S \ni t/s \cdot s = t$ meets $T$, so, since $S \mid T$, we must have $T \supseteq t/s \cdot S \ni t/s \cdot s' = (s'/s) \cdot t$. By swapping $s$ and $s'$, we get that $(s'/s)^{-1} \cdot t \in T$ as well. Repeating inductively, we get that $(s'/s)^k \cdot t \in T$ for arbitrary $k \in \bZ$ as desired.

    Now, let $a := \inf\{s'/s : s < s' \in S\}$. Then there is a sequence $a_n := s'_n/s_n \to a$ for some $s_n < s'_n \in S$. By the preceding claim, we have $T = \overline{T} \supseteq \overline{\bigcup_{n \in \bN} a_n^\bZ \ T}$. We claim that $a > 1$. Indeed, if $a = 1$, then $a_n^\bZ T$ forms a cover of $\bR^\times$ with maximum multiplicative distance $a_n \to 1$, so $\bigcup_{n = 1}^\infty a_n^\bZ T$ must be dense in $\bR^\times$, which forces $T = \bR^\times$, whereas we assumed otherwise. Therefore $a > 1$ and
    \begin{align*}
        T \supseteq \overline{\bigcup_{n \in \bN} a_n^\bZ T}
          = \overline{\bigcup_{k \in \bZ} \bigcup_{t \in T} \{a_n^k t : n \in \bN\}}
          \supseteq \bigcup_{k \in \bZ} \bigcup_{t \in T}\overline{\{a_n^k t : n \in \bN\}}
          \supseteq \bigcup_{k \in \bZ} \bigcup_{t \in T} \{a^k t\}
          = a^\bZ T.
    \end{align*}
    Finally, let $B := T \cap [1, a)$. For any $t \in T$, we can choose $k \in \bZ$ so that $a^k t \in [1, a)$ and, furthermore, $a^k t \in a^\bZ T \subseteq T$, so in fact $b := a^k t \in B$ and thus $t = a^{-k} a^k t = a^{-k} b \in a^\bZ B$. That is, $T \subseteq a^\bZ B \subseteq a^\bZ T \subseteq T$.
\end{proof}

We also require a slight variation on the usual characterization of convexity: a concave curve that passes above and then below a convex curve must remain below it.

\begin{lemma}
    \label{lemma:convex_vs_affine}
    Let $f, g : (a, b) \to \bR$ so that $f$ is convex and $g$ is concave. Then, if $f(x_1) \geq g(x_1)$ and $f(x_2) \leq g(x_2)$ for some $x_1 < x_2 \in (a, b)$, then we have $f(x) \geq g(x)$ for all $x \in (a, x_1)$.
\end{lemma}

\begin{proof}
    Suppose we could find $x \in (a, x_1)$ with $f(x) < g(x)$. Then, by convexity of $f$ and concavity of $g$ over $[x, x_2]$, we have
    \begin{align*}
        f(x_1)
            \leq \frac{x_2 - x_1}{x_2 - x}\underbrace{f(x)}_{< g(x)} + \frac{x_1 - x}{x_2 - x} \underbrace{f(x_2)}_{\leq g(x_2)}
            < \frac{x_2 - x_1}{x - x_2} g(x) + \frac{x_1 - x}{x - x_2} g(x_2)
            \leq g(x_1),
    \end{align*}
    which is a contradiction.
\end{proof}
% \pasin{Can't the above be directly seen from $f(x) - g(x)$ being convex?}

Now, our first important observation is that $h_i$ is supported at the boundary of its region of dominance, excluding simple crossings. This is essentially because convex and affine curves can only cross at points.

\begin{proposition}
    \label{prop:boundary_support}
    Let $h_1, h_2 : \bR^\times \to \bR$ be convex. Then
    \begin{enumerate}[(i)]
        \item $\partial X(h_1, h_2) \setminus C(h_1, h_2) \subseteq \supp(h_2)$; and
        \item $\partial X(h_2, h_1) \setminus C(h_1, h_2) \subseteq \supp(h_1)$.
    \end{enumerate}
\end{proposition}

\begin{proof}
    We just prove (i) as (ii) follows by symmetry. In this case it is equivalent to show that $\partial X(h_1, h_2) \setminus \supp(h_2) \subseteq C(h_1, h_2)$.
    
    To that end, consider any $x \in \partial X(h_1, h_2) \setminus \supp(h_2)$. Then there must be some interval $(x r^{-1}, x r)$ around $x$ on which $h_2$ is affine. Moreover, since $x \in \partial{X(h_1, h_2)}$, there must be some $x' \in (x r^{-1}, x r)$ for which $h_1(x') < h_2(x')$. On the other hand, both $X(h_1, h_2)$ and $X(h_2, h_1)$ are disjoint open subsets of a connected space, so neither meets $\partial{X(h_1, h_2)}$. In particular, $h_1(x) = h_2(x)$ and $x \neq x'$. Without loss of generality, $x r^{-1} < x' < x$.

    Now, we claim that $(x', x) \subseteq X(h_1, h_2)$ and $(x, x r) \subseteq X(h_2, h_1)$. Indeed, if there were $x'' \in (x', x)$ with $h_1(x'') \geq h_2(x'')$, then by applying Lemma~\ref{lemma:convex_vs_affine} with $x_1 = x''$ and $x_2 = x$, we get $h_1(x') \geq h_2(x')$, which is a contradiction. Likewise, if we could find $x'' \in (x, x r)$ with $h_1(x'') \leq h_2(x'')$, then applying Lemma~\ref{lemma:convex_vs_affine} to $x_1 = x$ and $x_2 = x''$ yields once again $h_1(x') \geq h_2(x')$, which is impossible. Overall, shrinking $r$ a little as needed, we have that $x \in C(h_1, h_2)$.
\end{proof}

The other key observation is that the support of $h_i$ meets every maximal interval in which it is dominated. Recall that every open set can be decomposed into connected components (disjoint open intervals in this case). For convenience, we will write $h(0) := \lim_{x \to 0} h(x)$ when this limit exists.

\begin{proposition}
    \label{prop:dominated_support}
    Let $h_1, h_2 : \bR^\times \to \bR$ be convex, bounded, and decreasing such that $h_1(0) = h_2(0)$. Then for any connected component $I \neq \bR^\times$ of $X(h_1, h_2)$, we have $I \cap \supp(h_1) \neq \varnothing$. Similarly for each connected component $I \neq \bR^\times$ of $X(h_2, h_1)$, we have $I \cap \supp(h_2) \neq \varnothing$.
\end{proposition}

\begin{proof}
    The two claims are symmetric so we just prove the first one. Let $I = (a, b) \neq \bR^\times$ be a connected component of $X(h_1, h_2)$ and assume that $I$ does not meet $\supp(h_1)$, i.e. $h_1$ is affine over the interval $\overline{I}$.
    
    First, we claim that $b < \infty$. Indeed, suppose that we had $b = \infty$. Then $a > 0$ as $I \neq \bR^\times$. Now, since $h_1$ and $h_2$ are convex and, in particular continuous over $\bR^\times$, if it were the case that $h_1(a) < h_2(a)$, then we could extend the interval $(a, \infty)$ to the left to obtain a larger interval in $X(h_1, h_2)$, whereas we assumed that $I$ was a connected component. Thus $h_2(a) \leq h_1(a)$. But $h_1$ is affine and bounded on an unbounded interval $[a, \infty)$, so the only possibility is that $h_1$ is constant on $[a, \infty)$. In particular since $h_2$ is decreasing, $h_2$ must be dominated by $h_1$ over all of $I = (a, \infty) \subseteq X(h_1, h_2)$, which is a contradiction.

    Next, we claim that $h_2$ is dominated by $h_1$ at the endpoints $\{a, b\}$. Indeed, if $a = 0$, then $h_1(a) = h_2(a)$ by assumption and, if $a > 0$, then by the same interval extension argument as before, we have $h_2(a) \leq h_1(a)$. Likewise, since $b < \infty$ and since the interval $I$ is maximal, we must have $h_2(b) \leq h_1(b)$ as well. Overall, $h_2$ is dominated at the endpoints of $I$ by an affine curve $h_1$, so by convexity $h_2$ must be dominated by $h_1$ over all of $I \subseteq X(h_1, h_2)$, which is impossible.
\end{proof}

We now can prove our main result of the section, which implies \Cref{thm:well_ordering}. We note the small caveat that the queries must be non-degenerate, namely their PLDs must be supported at at least two finite points. This is not serious practical limitation.

\begin{theorem}
    \label{thm:hockey_link}
    Let $h_1, h_2 : \bR^\times \to \bR$ be non-degenerate hockey-stick curves that are not well-ordered. Then one of $\supp(h_1), \supp(h_2)$ links $X(h_1, h_2)$ and $X(h_2, h_1)$.
\end{theorem}

\begin{proof}
    We begin with the easy case: Assume that one of $\partial X(h_1, h_2), \partial X(h_2, h_1)$ is not fully contained in $C(h_1, h_2)$. Without loss of generality, suppose $\partial X(h_1, h_2) \nsubseteq C(h_1, h_2)$. By Proposition~\ref{prop:boundary_support}, $\supp(h_2)$ meets $\partial X(h_1, h_2) \subseteq \overline{X(h_1, h_2)}$. Furthermore, since $h_1$ and $h_2$ are not well-ordered, $X(h_2, h_1) \neq \varnothing, \bR^\times$ and thus $\supp(h_2)$ meets $X(h_2, h_1)$ as well by Proposition~\ref{prop:dominated_support}. By Lemma~\ref{lemma:closure_linkage}, $\supp(h_2)$ must link $X(h_1, h_2)$ and $X(h_2, h_1)$.

    On the other hand, assume that $\partial X(h_1, h_2)$ and $\partial X(h_2, h_1)$ are both contained in $C(h_1, h_2)$ and assume toward a contradiction that neither $\supp(h_1)$ nor $\supp(h_2)$ links $X(h_1, h_2)$ and $X(h_2, h_1)$. We will derive a contradiction in two steps: First, we establish the structure of $E(h_1, h_2)$, $\supp(h_1)$, $\supp(h_2)$ and second, we derive a recurrence for the values of $h_1$ and $h_2$ at the points where they meet. By analyzing the characteristic polynomial of this recurrence, we can show that $h_1$ and $h_2$ must be either unbounded or constant, contradicting our assumptions.

    Now, our first major goal is to argue that $E(h_1, h_2)$ can be expressed of the form $a^\bZ \{b_1, b_2\}$ for some $1 \leq b_1 < b_2 < a$ and, exchanging the roles of $h_1$ and $h_2$ as needed, that $\supp(h_1) = a^\bZ s_1$ and $\supp(h_2) = a^\bZ s_2$ for some $s_1 \in (b_1, b_2)$ and $s_2 \in (b_2, ab_1)$.

    To this end, we first argue that $E(h_1, h_2)$ lives in both $\overline{X(h_1, h_2)}$ and $\overline{X(h_2, h_1)}$. Indeed, suppose we could find some $x \in \partial \textrm{int}(E(h_1, h_2))$. In this case,
    \begin{align*}
        x & \in \partial E(h_1, h_2) \\
            & = \partial (X(h_1, h_2) \cup X(h_2, h_1))^C \\
            & = \partial (X(h_1, h_2) \cup X(h_2, h_1)) \\
            & \subseteq \partial X(h_1, h_2) \cup \partial X(h_2, h_1) \\
            & \subseteq C(h_1, h_2),
    \end{align*}
    so, by construction of $C(h_1, h_2)$, there is some punctured neighbourhood $(x r^{-1}, x r) \setminus \{x\}$ of $x$ lying outside of $E(h_1, h_2) \supseteq \textrm{int}(E(h_1, h_2))$, which would contradict $x \in \partial \textrm{int}(E(h_1, h_2))$. Thus $\textrm{int}(E(h_1, h_2))$ has empty boundary, so it must be both closed and open in $\bR^\times$. But $h_1$ and $h_2$ cross, so we cannot have $\textrm{int}(E(h_1, h_2)) = \bR^\times$. By connectivity of $\bR^\times$, the only possibility is that $\textrm{int}(E(h_1, h_2)) = \varnothing$. In particular, by construction of the simple crossing points, we have
    \begin{align*}
        E(h_1, h_2)
            = \partial E(h_1, h_2)
            \subseteq C(h_1, h_2)
            \subseteq \partial X(h_1, h_2) \cap \partial X(h_2, h_1)
            \subseteq \overline{X(h_1, h_2)} \cap \overline{X(h_2, h_1)}.
    \end{align*}

    We can now deduce the structure of $E(h_1, h_2)$. First, notice that $\supp(h_i)$ divides $E(h_1, h_2)$. If not, then $\supp(h_i)$ must link $E(h_1, h_2)$ with $\bR^\times \setminus E(h_1, h_2) = X(h_1, h_2) \cup X(h_2, h_1)$, so it must either link $E(h_1, h_2) \subseteq \overline{X(h_2, h_1)}$ with $X(h_1, h_2)$ or it must link $E(h_1, h_2) \subseteq \overline{X(h_1, h_2)}$ with $X(h_2, h_1)$. In either case, by Lemma~\ref{lemma:closure_linkage}, $\supp(h_i)$ links $X(h_1, h_2)$ with $X(h_2, h_1)$, whereas we assumed otherwise. Therefore, $\supp(h_i)$ divides $E(h_1, h_2)$, so, by Proposition~\ref{prop:dense_subgroup}, there must be $a > 1$ and $B \subseteq [1, a)$ such that $E(h_1, h_2) = a^\bZ B$ and such that the elements of $\supp(h_1)$ and of $\supp(h_2)$ are multiplicatively separated by at least a factor of $a$. We claim that $B$ contains no more than two points. Indeed, suppose we could find $b_1 < b_2 < b_3 \in B$. By construction of $C(h_1, h_2) \ni b_1, b_2, b_3$, $(b_1, b_2) \cup (b_2, b_3) \cup (b_3, ab_1)$ contains at least three connected components of $X(h_1, h_2) \cup X(h_2, h_1)$, so in particular, it must contain at least two connected components of $X(h_1, h_2)$ or of $X(h_2, h_1)$. By Proposition~\ref{prop:dominated_support}, $(b_1, ab_1)$ contains at least two points from $\supp(h_1)$ or from $\supp(h_2)$, which is impossible because these points were multiplicatively separated by $a$. Thus $|B| \leq 2$. On the other hand, if $B = \{b\}$, then we have $E(h_1, h_2) = a^\bZ\{b\} = a^{2\bZ}\{b\} \cup a^{2\bZ + 1}\{b\} = (a^2)^\bZ\{b, ab\}$, so in all cases we can write $E(h_1, h_2) = a^\bZ\{b_1, b_2\}$ for some $1 \leq b_1 < b_2 < a$.

    Finally, we determine the structure of $\supp(h_1)$ and $\supp(h_2)$. For $k \in \bZ$, consider the alternating intervals $I_k := (a^k b_1, a^k b_2)$ and $J_k := (a^k b_2, a^{k+1} b_1)$. Then, for every $k \in \bZ$, $I_k$ does not meet $E(h_1, h_2)$, so, by the intermediate value theorem, either $I_k \subseteq X(h_1, h_2)$ or $I_k \subseteq X(h_2, h_1)$ and, similarly, either $J_k \subseteq X(h_1, h_2)$ or $J_k \subseteq X(h_2, h_1)$. Now, assume without loss of generality that $I_0 \subseteq X(h_1, h_2)$. By induction and by construction of the simple crossings $C(h_1, h_2) \supseteq E(h_1, h_2)$, this forces $I_k \subseteq X(h_1, h_2)$ and $J_k \subseteq X(h_2, h_1)$ for every $k \in \bZ$. By Proposition~\ref{prop:dominated_support}, we can find $s_1 \in (b_1, b_2) \cap \supp(h_1)$ and $s_2 \in (b_2, ab_1) \cap \supp(h_2)$. In particular, $b_1/s_1 \cdot \supp(h_1) \ni b_1$ meets $E(h_1, h_2) = a^{\bZ}\{b_1, b_2\} \ni b_1$, so, since $\supp(h_1) \mid E(h_1, h_2)$, we must have that $b_1/s_1 \cdot \supp(h_1) \subseteq a^{\bZ} \{b_1, b_2\}$, i.e. $\supp(h_1) \subseteq a^{\bZ} \{s_1, b_2/b_1 \cdot s_1\}$. We now eliminate $b_2/b_1 \cdot s_1$ from the residue set. To that end, suppose we could find $s'_1 := a^k b_2/b_1 \cdot s_1 \in \supp(h_1)$. Then $\frac{b_1^2}{s_1 b_2} \cdot s'_1 = a^k b_1 \in E(h_1, h_1)$, so, since $\supp(h_1) \mid E(h_1, h_2)$, we must have $\frac{b_1^2}{b_2} = \frac{b_1^2}{s_1 b_2} \cdot s_1 \in E(h_1, h_2) = a^\bZ \{b_1, b_2\}$ as well. But, $b_1/b_2 \in (a^{-1}, 1)$, so $\frac{b_1^2}{b_2} = a^k b_1$ is impossible, whereas $\frac{b_1^2}{b_2} = a^k b_2$ forces $b_1/b_2 = a^{-1/2}$. Moreover, $s_1 \in (b_1, b_2)$, so we have
    \begin{align*}
        s'_1
            = a^k b_2/b_1 \cdot s_1
            \in (a^k b_2/b_1 \cdot b_1, a^k b_2/b_1 \cdot b_2)
            = (a^k b_2, a^k b_1 \cdot (b_2/b_1)^2)
            = (a^k b_2, a^{k + 1} b_1)
            = J_k,
    \end{align*}
    which contradicts our assumption that $\supp(h_1)$ fails to link $X(h_1, h_2) \supseteq I_0$ with $X(h_2, h_1) \supseteq J_k$. Therefore $\supp(h_1) \subseteq a^\bZ s_1$ which, noting that $a^\bZ s_1 \cap I_k = a^k s_1$, forces $\supp(h_1) = a^\bZ s_1$ due to Proposition~\ref{prop:dominated_support}. By a symmetric argument, $\supp(h_2) = a^\bZ s_2$ as well.

    Now that we understand the structure of the crossings and of the support, we can calculate the trajectory of $h_1$ and $h_2$ over the crossing points. To this end, consider $y_k := h_1(a^k b_1) = h_2(a^k b_1)$ and $y'_k := h_1(a^k b_2) = h_2(a^k b_2)$. Now, for any $k \in \bZ$, the interval $[a^k b_1, a^{k+1} b_2] = \overline{I_k} \cup \overline{J_k} \cup \overline{I_{k + 1}}$ meets $\supp(h_2)$ at $a^k s_2$ and nowhere else. Thus $h_2$ is affine over $[a^k b_1, a^k s_2]$ and $[a^k s_2, a^{k+1}b_2]$ separately. By extrapolation in the former interval, we get
    \begin{align*}
        h_2(a^k s_2)
            = \frac{a^k b_2 - a^k s_2}{a^k b_2 - a^k b_1}h_1(a^k b_1) +
              \frac{a^k s_2 - a^k b_1}{a^k b_2 - a^k b_1}h_1(a^k b_2)
            = \frac{b_2 - s_2}{b_2 - b_1}y_k + \frac{s_2 - b_1}{b_2 - b_1}y'_k
    \end{align*}
    and, furthermore, extrapolation over the latter interval yields
    \begin{align*}
        y'_{k + 1}
            & = h_2(a^{k+1}b_2) \\
            & = \frac{a^{k+1}b_1 - a^{k+1}b_2}{a^{k+1}b_1 - a^k s_2}h_1(a^k s_2) +
                \frac{a^{k+1}b_2 - a^k s_2}{a^{k+1}b_1 - a^k s_2}h_2(a^{k+1}b_1) \\
            & = \frac{a(b_1 - b_2)}{ab_1 - s_2}\left(
                    \frac{b_2 - s_2}{b_2 - b_1}y_k + \frac{s_2 - b_1}{b_2 - b_1}y'_k
                \right) +
                \frac{ab_2 - s_2}{ab_1 - s_2}y_{k + 1} \\
            & = \underbrace{\frac{a(s_2 - b_2)}{ab_1 - s_2}}_{=: U'}y_k + \underbrace{-\frac{a(s_2 - b_1)}{ab_1 - s_2}}_{=: V'}y'_k + \underbrace{\frac{ab_2 - s_2}{ab_1 - s_2}}_{=: W'}y_{k + 1}.
    \end{align*}
    % the interval $[a^k b_2, a^{k+2} b_1] = \overline{J_k} \cup \overline{I_{k + 1}} \cup \overline{J_{k + 1}}$ meets $\supp(h_1)$ at $a^{k + 1} s_1$ and nowhere else. Thus $h_1$ is affine over $[a^k b_2, a^{k+1}s_1]$ and $[a^{k+1}s_1, a^{k+2}b_1]$ separately. By extrapolation in the former interval, we get
    % \begin{align*}
    %     h_1(a^{k+1}s_1)
    %         = \frac{a^{k+1}b_1 - a^{k+1}s_1}{a^{k+1}b_1 - a^k b_2}h_1(a^k b_2) +
    %           \frac{a^{k+1}s_1 - a^k b_2}{a^{k+1}b_1 - a^k b_2}h_1(a^{k+1}b_1)
    %         = \frac{a(b_1 - s_1)}{ab_1 - b_2}y'_k + \frac{as_1 - b_2}{ab_1 - b_2}y_{k + 1}
    % \end{align*}
    % and, furthermore, extrapolation over the latter interval yields
    % \begin{align*}
    %     y_{k + 2}
    %         & = h_1(a^{k+2}b_1) \\
    %         & = \frac{a^{k+1}b_2 - a^{k+2}b_1}{a^{k+1}b_2 - a^{k+1} s_1}h_1(a^{k+1} s_1) +
    %             \frac{a^{k+2}b_1 - a^{k+1} s_1}{a^{k+1}b_2 - a^{k+1} s_1}h_1(a^{k+1}b_2) \\
    %         & = \frac{b_2 - ab_1}{b_2 - s_1}\left(
    %                 \frac{a(b_1 - s_1)}{ab_1 - b_2}y'_k + \frac{as_1 - b_2}{ab_1 - b_2}y_{k + 1}
    %             \right) +
    %             \frac{ab_1 - s_1}{b_2 - s_1}y'_{k + 1} \\
    %         & = \underbrace{\frac{a(s_1 - b_1)}{b_2 - s_1}}_{=: U}y'_k + \underbrace{-\frac{as_1 - b_2}{b_2 - s_1}}_{=: V}y_{k + 1} + \underbrace{\frac{ab_1 - s_1}{b_2 - s_1}}_{=: W}y'_{k + 1}.
    % \end{align*}
    By an analogous argument, we have that
    \begin{align*}
    y_{k + 2}
        = \underbrace{\frac{a(s_1 - b_1)}{b_2 - s_1}}_{=: U}y'_k + \underbrace{-\frac{as_1 - b_2}{b_2 - s_1}}_{=: V}y_{k + 1} + \underbrace{\frac{ab_1 - s_1}{b_2 - s_1}}_{=: W}y'_{k + 1}
    \end{align*}
    as well, which together yields the system
    % Analogously, we have $y_{k + 3} = U_d y_k + V_d y_{k+1} + W_d y_{k+2}$ for even $k$. Together, we have the system
    \begin{align*}
        \left\{
        \begin{aligned}
            y'_{k + 1} & = U' y_k + V' y'_k + W' y_{k + 1} \\
            y_{k + 2} & = U y'_k + V y_{k+1} + W y'_{k+1} \\
            y'_{k + 2} & = U' y_{k+1} + V' y'_{k+1} + W' y_{k+2} \\
            y_{k + 3} & = U y'_{k+1} + V y_{k+2} + W y'_{k+2}
        \end{aligned}
        \right.
    \end{align*}
    for $k \in \bZ$. By isolating $y'_{k+1}$ in terms of $y_k, y_{k+1}, y_{k+2}$ from the first two equations, substituting into the equation for $y'_{k+2}$, and finally substituting this into the equation for $y_{k + 3}$, we obtain the symmetrized recurrence
    \begin{align*}
        y_{k + 3} = U U' y_k + (U W' + U' W - V V') y_{k + 1} + (V + V' + W W') y_{k + 2}
    \end{align*}
    for $k \in \bZ$. Thus $(y_k)_{k \in \bZ}$ satisfies a linear homogeneous recurrence with characteristic polynomial
    \begin{align*}
        f(t) = t^3 - (V + V' + W W')t^2 - (U W' + U' W - V V') t - U U',
    \end{align*}
    which after applying some computational effort factors as
    \begin{align*}
        f(t) = (t - 1)(t - a)\left( t - a\left( \frac{s_1 - b_1}{b_2 - s_1} \right)\left( \frac{s_2 - b_2}{ab_1 - s_2} \right) \right).
    \end{align*}
    In particular, every root of $f$ lies in $\bR^\times$. On the other hand, a homogeneous linear recurrence over the integers admits a bounded non-constant solution if and only its characteristic polynomial has a non-unit root lying on the complex unit circle (see e.g. Section 2.3 of \cite{elaydi2005introduction}). Therefore either $y_k = h_1(a^k b_1) = h_2(a^k b_1)$ is unbounded, which is impossible because hockey-stick curves are bounded, or it must be constant, which is also impossible because hockey-stick curves are monotone, so this would force $h_1$ and $h_2$ to be constant and thus well-ordered.
\end{proof}

\Cref{thm:well_ordering} now follows immediately by combining Propositions~\ref{prop:pld_gap_amplification} and~\ref{prop:support_conversion} with Theorem~\ref{thm:hockey_link} by taking $B := \sup\{B_1, B_2\}$.
% \todo{elaborate a little maybe}

	% \input{sections/special_cases_app}
\end{document}
\section{Proof of \Cref{lemma:restricted_adversary}}
\label{app:restricted_adversary_proof}

\begin{proof}
    The case where $k = 1$ is immediate, so we assume $k > 1$. We first show that
    \begin{align*}
        \pld(\Gamma) \preceq \sup\{L \oplus \pld(\Gamma_L) : L \in \Gamma(\lambda)\}.
    \end{align*}
    To that end, consider any $\Gamma$-adversary $M$. Decompose $M$ as the adaptive composition of $M_1(\cdot)$ with some $M'(\cdot; y_1)$ and let $L := \pld(M_1) \in \Gamma(\lambda)$. Then, clearly, for every fixed $y_1$, $M'(\cdot; y_1)$ is a $\Gamma_L$-adversary and, in particular, $\pld(M'(\cdot; y_1)) \preceq \pld(\Gamma_L)$. By \Cref{prop:composition_convolution}, we have that $\pld(M) \preceq L \otimes \pld(\Gamma_L)$ and the desired inequality follows by taking suprema.

    More surprising is the reverse inequality, which we prove constructively by passing to hockey-stick curves. Let $L \in \Gamma(\lambda)$, let $x \in \bR^\times$, and let $\gamma > 0$. Choose any mechanism $M_1$ that has PLD $L$ and let $\ell_1 := \frac{dM_1(D_1)}{dM_1(D_2)}$ denote the corresponding likelihood function. Moreover, for any $x \in \R^\times$ we can choose by \Cref{prop:sup-conv}, a $\Gamma_L$-adversary $M^x$ such that
    \begin{align*}
        H_x(M^x) \geq \sup_{M' \in \Adv(\Gamma_L)}H_x(M') - \gamma.
    \end{align*}
    Now consider the $\Gamma$-adversary $M := M_1 \otimes M_2$ where $M_2(D; y_1) := M^{x/\ell_1(y_1)}(D)$. By \Cref{prop:hockey_stick_composition}, we get
    \begin{align*}
        h_M(x)
            & = \E_{Y_1 \sim M_1(D_1)}[h_{M_{x/\ell_1(Y_1)}}(x/\ell_1(Y_1))] \\
            & \geq \E_{Y_1 \sim M_1(D_1)}[\sup_{M' \in \Adv(\Gamma_L)}h_{M'}(x/\ell_1(Y_1)) - \gamma] \\
            & = (h_{M_1} \otimes \sup_{M' \in \Adv(\Gamma_L)} h_{M'})(x) - \gamma.
    \end{align*}
    Again, taking suprema, we get that
    \begin{align*}
        \sup_{M \in \Adv(\Gamma)} h_M \succeq h_{M_1} \otimes \sup_{M' \in \Adv(\Gamma_L)} h_{M'},
    \end{align*}
    and thus $\pld(\Gamma) \succeq L \otimes \pld(\Gamma_L)$, which completes the proof since $L$ was arbitrary.
\end{proof}
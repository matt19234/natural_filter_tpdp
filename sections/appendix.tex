\appendix

\section{$f$-DP Results, Proofs and Special Cases}

We present additional results focusing on $f$-DP, which we use to instantiate our main results \Cref{thm:free_natural_filter,thm:universal_free_natural_filter,thm:well_ordering} on special cases, that may help with intuition. For completeness, \Cref{alg:fdp_natural_filter} states the privacy filter algorithm for $f$-DP, and is a direct analogy to \Cref{alg:natural_filter}.

\begin{algorithm}[tb]
    \caption{NaturalFilter ($f$-DP)}
    \label{alg:fdp_natural_filter}
    \begin{algorithmic}
        \REQUIRE Privacy bound $f$, adversary (or analyst) $\cA$, family of tradeoff curves $\cF$, query capacity $k$
        \STATE $i=0$
        \WHILE{$\cA$ continues and $i < k$}
        \STATE $i \leftarrow i+1$
        \STATE $\cA$ gives mechanism $\cM_i$ that is $f_i$-DP, with $f_i \in \cF$ \algcomment{can depend on previous results $Y_{<i}$}
        \IF{$f_1 \otimes \ldots \otimes f_i \geq f$}
        \STATE $\cA$ receives $Y_i \sim \cM_i$
        \ENDIF
        \ENDWHILE
        \RETURN $(Y_1, \ldots, Y_i)$
    \end{algorithmic}
\end{algorithm}


\subsection{Proof of \Cref{prop:piecewise-linear-composition}}\label{sec:proof-piecewise-linear}

% \begin{proposition}[Exact composition of piecewise-linear tradeoff curves]
% \label{prop:piecewise-linear-composition}
%     Consider two piecewise-linear tradeoff curves $f_1, f_2$. 
%     Denote $\alpha_{i, n_i} \in [0,1]$ the lengths of intervals over which $f_i$ is linear, 
%     and $f'_{i,n_i} \leq 0$ the slope of $f_i$ on those intervals. 
%     Denote $\delta_i \triangleq 1-f_i(0)$.
    
%     That is, $\sum_{j=1}^{n_i} \alpha_{i, j} = 1$, and, noting $m_i(\alpha) = \arg\max_{n: \in \{1, n_i\}} \sum_{j=1}^{n} \alpha_{i,j} \leq \alpha$, we have:
%     \[
%         f_i(\alpha) = 1 - \delta_i + \sum_{j=1}^{m_i(\alpha)} f'_{i,j} \alpha_{i,j} + 
%             \big(\alpha - \sum_{j=1}^{m_i(\alpha)} \alpha_{i,j} \big) f'_{i, j+1}.
%     \]

%     Let $k \in [1, n_1 n_2]$ index all combinations of $(j_{1, k}, j_{2,k}) \in [1, n_1] \times[1, n_2]$ 
%     in order of increasing values of $-f'_{1,j_{1,k}}f'_{2,j_{2,k}}$. 
%     That is, $k$ indexes all products of a slope from $f_1$ and a slope for $f_2$ 
%     (which are now positive) sorted in decreasing order of magnitude. 
%     With ne minus sign, the first index is thus the steepest negative slope.
    
%     Then the composition $f \triangleq f_1 \otimes f_2$ is also a piecewise linear tradeoff curve, 
%     with $\delta = \delta_1 + \delta_2 - \delta_1\delta_2$, and $\alpha_k = \alpha_{1,k}\alpha_{2,k}$ 
%     and slopes $f'_k = -f'_{1,j_{1,k}}f'_{2,j_{2,k}}$. 
%     That is, denoting $m(\alpha) = \arg\max_{k: \in \{1, n_1n_2\}} \sum_{k=1}^{n} \alpha_k \leq \alpha$, 
%     $f = f_1 \otimes f_2$ is such that:
%     \[
%         f(\alpha) = 1 -\delta_1 - \delta_2 + \delta_1\delta_2 + 
%             \sum_{j=1}^{m(\alpha)} f'_k \alpha_k + 
%             \big(\alpha - \sum_{j=1}^{m(\alpha)} \alpha_k \big) f'_{k+1}.
%     \]
% \end{proposition}
\begin{proof}
For each $i \in \{1,2\}$, let $f_i$ be a tradeoff curve with $n_i$ linear segments and $f_i(0) =1 -\delta_i$.  Each linear segment $j \in [n_i]$ has a slope $f'_{i,j} \le 0$ over an interval $[\alpha_{i,j}, \alpha_{i,j+1})$.
%the last linear segment has a slope $f'_{i,n_i}$ over an interval $[\alpha_{i,n_i}, \alpha_{i,n_i+1}]$. 
The width of   interval $j$ is  denoted by $w_{i,j} $. We also assume $-f'_{i,1} \ge -f'_{i,2 }\ge \cdots \ge -f'_{i,n_i } $.
Note that we have $\alpha_{i,1} = 0$, $\alpha_{i,n_i +1} = 1$, $f_i(1) = 0$, and $\sum_{j =1}^{n_i}w_{i,j} = 1$.
%We also have $\bigcup_{j=1}^{n_i} [\alpha_{i,j}, \alpha_{i,j+1}) \bigcup \{1 \} = [0,1]$. 

For each $i \in \{1,2 \}$, we construct a probability distribution $Q_i$ supported by $[0,1]$ with density function as follows:
\begin{equation}
     q_i(x) = \begin{cases}
      -f'_{i,1}, & \text{if } x \in [\alpha_{i,1}, \alpha_{i,2}),\\
       -f'_{i,2}, & \text{if }  x \in [\alpha_{i,2}, \alpha_{i,3}), \\
        \vdots & \\
       -f'_{i,j}, & \text{if }  x \in [\alpha_{i,j}, \alpha_{i,j+1}), \\
        \vdots & \\
       -f'_{i,n_i}, & \text{if }  x \in [\alpha_{i,n_i}, \alpha_{i,n_i+1}),\\
       (1-f_i(0)) \cdot \delta(x-1),  & \text{if }  x = \alpha_{i,n_i+1}.
    
%       (1-f_i(0)) \cdot \delta(x-1), & \text{if } x =1.
    \end{cases} 
\end{equation}
% where $P_i(\{1 \}) = 1-f_i(0)$. 
%Basically, all $w_{i,j}$ form a partition of $[0,1]$. % i.e., $\bigcup_{j \in [n_i]} \alpha_{i,j} = [0,1]$. 
%Let $P_i$ be a distribution with a point $w \in [\alpha_{i,j}, \alpha_{i,j+1})$ using  $-f'_{i,j}$ as its density when $j \in [n_i -1]$, and a point $w \in [\alpha_{i,n_i}, 1]$ using  $-f'_{i,n_i}$ as its density.
Let $P$ be the uniform distribution over $[0,1]$. Now, we show that $T(P,Q_i) = f_i$.

When $\alpha \in [\alpha_{i,1}, \alpha_{i,2})$, the optimal rejection rule is
\begin{equation*}
     \phi(x) = \begin{cases}
      \alpha/w_{i,1}, & \text{if }  x \in [\alpha_{i,1}, \alpha_{i,2}),\\
      1, & \text{if } x = 1, \\
      0, & \text{otherwise}.
    \end{cases} 
\end{equation*}
Note that the type I error is $\mathbb{E}_P[\phi] = \alpha = c \cdot w_{i,1}$. The type II error is 
\begin{equation*}
    \begin{array}{l}
1- \mathbb{E}_{Q_i}[\phi] =  1 -c \cdot w_{i,1} \cdot (-f'_{i,1}) -  (1-f_i(0)) = f_i(0) +\alpha f'_{i,1} = f_i(\alpha)= T(P,Q_i)(\alpha).
    \end{array}
\end{equation*}

When $\alpha \in [\alpha_{i,2}, \alpha_{i,3})$, the optimal rejection rule is 
\begin{equation*}
     \phi(x) = \begin{cases}
      1, & \text{if }  x \in [\alpha_{i,1}, \alpha_{i,2}) \bigcup \{1\},\\
         c, & \text{if }  x \in [\alpha_{i,2}, \alpha_{i,3}),\\
      0, & \text{otherwise}.
    \end{cases} 
\end{equation*}
Note that the type I error is $\mathbb{E}_P[\phi] = \alpha = w_{i,1} + c \cdot w_{i,2}$.  The type II error is 
\begin{equation*}
    \begin{array}{l}
1- \mathbb{E}_{Q_i}[\phi] =  f_i(0) -     w_{i,1} \cdot (-f'_{i,1}) - c \cdot w_{i,2} \cdot (-f'_{i,2}) = f_i(0)+w_{i,1} \cdot  f'_{i,1} + (\alpha - w_{i,1})  f'_{i,2}  = f_i(\alpha)=T(P,Q_i)(\alpha).
    \end{array}
\end{equation*}

More generally, when $\alpha \in [\alpha_{i,j}, \alpha_{i,j+1})$, the optimal rejection rule is 
\begin{equation*}
     \phi(x) = \begin{cases}
      1, & \text{if }  x \in [\alpha_{i,1}, \alpha_{i,j-1}) \bigcup \{1 \},\\
         c, & \text{if }  x \in [\alpha_{i,j}, \alpha_{i,j+1}),\\
      0, & \text{otherwise}.
    \end{cases} 
\end{equation*}
Note that the type I error is $\mathbb{E}_P[\phi] = \alpha = \sum_{s=1}^{j-1} w_{i,s} + c \cdot w_{i,j}$. The type II error is 
\begin{equation*}
    \begin{array}{lll}
1- \mathbb{E}_{Q_i}[\phi] &= & f_i(0) -     \sum_{s=1}^{j-1}w_{i,s} \cdot (-f'_{i,s}) - c \cdot w_{i,j} \cdot (-f'_{i,j})\\
& = & f_i(0)+\sum_{s=1}^{j-1}w_{i,s} \cdot  f'_{i,s} + (\alpha -\sum_{s=1}^{j-1} w_{i,s})  f'_{i,j}  = f_i(\alpha)=T(P,Q_i)(\alpha).
    \end{array}
\end{equation*}










Now, we construct a testing problem $P \times P \text{ vs } Q_1 \times Q_2 $. 
%{\color{red} we need to make sure the sum of all densities is 1, i.e., $\sum_{j \in n_i} w_{i,j} \cdot (-f'_{i,j}) =  $.}
We first compute $(-f'_{1,k_1}) \cdot (-f'_{2,k_2})$ for all $k_1 \in [n_1]$ and $k_2 \in [n_2]$. Then, we sort all these $n_1 \cdot n_2$ values  in a non-increasing order, using $k \in [n_1 \cdot n_2]$ as the index. For a fixed $k$, we know  its respective $k_1, k_2$ and then $w_{1, k_1}, w_{2, k_2}$ and $f'_{1, k_1}, f'_{2, k_2}$. %Let %$r_k := w_{1, k_1} \cdot w_{2, k_2} \cdot f'_{1, k_1} \cdot f'_{2, k_2}$. 
Let $\tilde{w}_k := w_{1, k_1} \cdot w_{2, k_2}$ and  $\tilde{f}'_k := f'_{1, k_1} \cdot f'_{2, k_2}$.









% To sum up, when $\alpha \in \left[\sum_{j=1}^{s} \tilde{f}'_j \cdot \tilde{w}_j , \ \sum_{j=1}^{s+1} \tilde{f}'_j \cdot \tilde{w}_j \right]$ where $s \in \{1,2, \dotsc, n_1 \cdot n_2 -1\}$, we have

%   \begin{equation*}
%         T(P_1 \times P_2, Q \times Q) (\alpha) =  1- \sum_{j =1}^{s} \tilde{w}_j - \frac{\alpha - \sum_{j=1}^s \tilde{w}_j \cdot \tilde{f}'_j }{\tilde{f}'_{s+1}}. 
%     \end{equation*}
    

%The support of $P_i$ is $\bigcup_{j=1}^{n_i} w_{i,j} \bigcup \{1\}$.

\begin{enumerate}
    \item When $\alpha \in \left[0, \ \tilde{w}_1 \right)$, we have the optimal rejection rule is 
    \begin{equation*}
     \phi(x) = \begin{cases}
      1, & \text{if }  (x_1, x_2) \in [0, 1) \times \{1\} \bigcup \{1\} \times [0, 1) \bigcup \{1\} \times \{1\} \\
    
         c, & \text{if }  (x_1, x_2) \in [\alpha_{1,1}, \alpha_{1,2}) \times [\alpha_{2,1}, \alpha_{2,2})  \\
      0, & \text{otherwise}.
    \end{cases} 
\end{equation*}
The type I error is $\mathbb{E}_{P \times P}[\phi] = \alpha = c \cdot \tilde{w}_{1}$. The type II error is
\begin{equation*}
   \mathbb{E}_{Q_1 \times Q_2}[\phi] = \delta_1 + \delta_2 + \delta_1 \cdot  \delta_2 +
        c \cdot \tilde{w}_{1} \cdot \tilde{f}'_{1}   =  \delta_1 - \delta_2 - \delta_1   \delta_2 -\alpha \cdot \tilde{f}'_{1}.
\end{equation*}
Therefore, we have
\begin{equation*}
    \begin{array}{ll}
    
      &  T(P \times P, Q_1 \times Q_2) (\alpha) = 1- \mathbb{E}_{Q_1 \times Q_2}[\phi] =  1 - \delta_1 - \delta_2 - \delta_1   \delta_2 -\alpha \cdot \tilde{f}'_{1}.
            \end{array}
    \end{equation*}

    \item When $\alpha \in \left[\tilde{w}_{1} , \ \tilde{w}_1 + \tilde{w}_{2} \right)$, we have
    \begin{equation*}
        \begin{array}{l}
             \alpha = \tilde{w}_{1}  + c \cdot  \tilde{w}_{2} \quad \Rightarrow c \cdot  \tilde{w}_{2}  = \alpha - \tilde{w}_1,  \\
             T(P \times P, Q_1 \times Q_2) (\alpha)  =  1 - \delta_1 - \delta_2 - \delta_1   \delta_2- \tilde{w}_1 \cdot \tilde{f}'_1- c \cdot \tilde{w}_2 \cdot \tilde{f}'_2 \\
             =  1 - \delta_1 - \delta_2 - \delta_1   \delta_2- \tilde{w}_1 \cdot \tilde{f}'_1- (\alpha - \tilde{w}_1) \cdot \tilde{f}'_2.
        \end{array}
    \end{equation*}
    \item When $\alpha \in \left[ \tilde{w}_1 + \tilde{w}_2, \  \tilde{w}_1 + \tilde{w}_2 + \tilde{w}_3  \right)$, we have
    \begin{equation*}
        \begin{array}{l}
             \alpha = \sum_{s = 1}^{2}\tilde{w}_s + c \cdot \tilde{w}_3 \quad \Rightarrow c \cdot \tilde{w}_3 =  \alpha - \sum_{s = 1}^{2}\tilde{w}_s, \\
           T(P \times P, Q_1 \times Q_2) (\alpha) = 1 - \delta_1 - \delta_2 - \delta_1   \delta_2- \sum_{s =1}^{2} \tilde{w}_s \cdot \tilde{f}'_{s} - c \cdot \tilde{w}_3 \cdot \tilde{f}'_3 \\ =  1 - \delta_1 - \delta_2 - \delta_1   \delta_2- \sum_{s =1}^{2} \tilde{w}_s \cdot \tilde{f}'_s - (\alpha - \sum_{s=1}^2 \tilde{w}_s) \cdot   \tilde{f}'_3. 
        \end{array}
    \end{equation*}

    \item When $\alpha \in \left[\sum_{s=1}^{j}  \tilde{w}_s , \ \sum_{s=1}^{j+1} \tilde{w}_s \right)$,  we have
    \begin{equation*}
    \begin{array}{l}
      \alpha = \sum_{s = 1}^{j}\tilde{w}_s + c \cdot \tilde{w}_{j+1} \quad \Rightarrow c \cdot \tilde{w}_{j+1} =  \alpha - \sum_{s = 1}^{j}\tilde{w}_s, \\
     T(P \times P, Q_1 \times Q_2) (\alpha) = 1- \delta_1 - \delta_2 - \delta_1   \delta_2- \sum_{s =1}^{j} \tilde{w}_s \cdot \tilde{f}'_{s} - c \cdot \tilde{w}_{j+1} \cdot \tilde{f}'_{j+1} \\ =  1- \delta_1 - \delta_2 - \delta_1   \delta_2- \sum_{s =1}^{j} \tilde{w}_s \cdot \tilde{f}'_s - (\alpha - \sum_{s=1}^j \tilde{w}_s) \cdot   \tilde{f}'_{j+1}.   \end{array}
    \end{equation*}
\end{enumerate}
    \end{proof}



\subsection{A Simple Derivation of \Cref{thm:well_ordering} for Approximate DP}
\label{sec:appendix:well-ordered-eps-delt}

\begin{definition}[Crossing tradeoff curves]\label{def:crossing}
    We say that two tradeoff curves $f, g \in \cT$ cross if there exists $\alpha_1, \alpha_2 \in [0, 1]$ s.t.:
    \[
        f(\alpha_1) > g(\alpha_1) \ \textrm{and} \ f(\alpha_2) < g(\alpha_2) .
    \]
\end{definition}

By continuity, there has to be $\alpha \in (\alpha_1, \alpha_2)$ s.t. $f(\alpha) = g(\alpha)$.
For any two $(\epsilon, \delta)$-DP with tradeoff curves that cross, there cannot be free $f$-DP filters.

We can now prove \Cref{prop:no-commutativity-for-eps-delta}, a special case of \Cref{thm:well_ordering} for $(\epsilon,\delta)$-DP.

\begin{proof} 
    Consider $g_1\triangleq f_{\epsilon_1, \delta_1}$ and $g_2 \triangleq f_{\epsilon_2, \delta_2}$ that cross. By symmetry w.r.t. the $y=x$ axis, they have to cross at an $\alpha < \min\{ \alpha_1^\star, \alpha_2^\star  \}$, where $\alpha_i^\star \ \textrm{s.t.} \  \alpha_i^\star = g_i(\alpha_i^\star)$ (otherwise they would never cross).
    This means that their linear segments cross: one function starts strictly lower, and then ends-up strictly higher by $\min\{ \alpha_1^\star, \alpha_2^\star  \}$ (again, otherwise they would never cross). Without loss of generality, we reorder functions so that $g_2$ starts strictly lower.

    We now know that $\delta_2 > \delta_1 \geq 0$ ($g_2$ starts lower); $\epsilon_1 > \epsilon_2 \geq 0$ ($g_1$ has a steeper slope to catch-up to $g_2$ so that they can cross); and $\alpha_2^\star > \alpha_1^\star$. The last inequality holds because $g_1$ is strictly lower by the time is crosses $y=x$ at $\alpha_1^\star$, so at that point $g_2(\alpha_1^\star) > \alpha_1^\star$, and by continuity of $g_2$ we have that $\alpha_2^\star > \alpha_1^\star$.
    Since we are dealing with approximate DP curves, we also know that $\alpha_i^\star = \frac{1 - \delta_i}{1 + e^{\epsilon_i}}$.
    
    Now, in \Cref{thm:universal_free_natural_filter}, pick $\cH = \{g_1, g_2 \} \subseteq \cF$ and $g = g_1 \in \cH \subseteq \cF$.
    We make a few useful observations about the first segment of $g_1 \otimes g_1$, $g_1 \otimes g_2$, and $g_1 \otimes \conv(g_1, g_2)$, by applying \Cref{prop:piecewise-linear-composition}:
    
    \noindent The first segment of $g_1 \otimes g_1$ starts at $(1-\delta_1)^2$, with slope $-e^{2\epsilon_1}$ for $\alpha \in [0, \alpha_1^{\star2}]$. The second slope is $-e^{\epsilon_1}e^{-\epsilon_1}=-1$.
    
    \noindent The first segment of $g_1 \otimes g_2$ starts at $(1-\delta_1)(1-\delta_2)$, with slope $-e^{\epsilon_1+\epsilon_2}$ for $\alpha \in [0, \alpha_1^\star\alpha_2^\star]$. The second slope is $-e^{\epsilon_1-\epsilon_2}<-1$.

    \noindent The first segment of $\conv(g_1, g_2)$ starts at $(1-\delta_2)$ (the lowest point at $\alpha=0$), and since $g_1$ and $g_2$ cross, it meets $g_1$ at $\alpha = \alpha_1^\star$, where its value is $g_1(\alpha_1^\star) = 1 - \delta_1 - e^{\epsilon_1}\alpha_1^\star$. As a result, the first slope of $\conv(g_1, g_2)$ is $\frac{1 - \delta_1 - e^{\epsilon_1}\alpha_1^\star - (1-\delta_2)}{\alpha_1^\star} = \frac{\delta_2 - \delta_1}{\alpha_1^\star} - e^{\epsilon_1}$.
    
    \noindent The first segment of $g_1 \otimes \conv(g_1, g_2)$ starts at $(1-\delta_1)(1-\delta_2)$, with first slope $\frac{\delta_1 - \delta_2}{\alpha_1^\star}e^{\epsilon_1} - e^{2\epsilon_1}$ for $\alpha \in [0, \alpha_1^{\star2}]$.
    

    Let us study the first segment of $\conv(g_1 \otimes g_1, g_1 \otimes g_2)$ and compare it to that of $g_1 \otimes \conv(g_1, g_2)$. We first reason about the possible shapes of $\conv(g_1 \otimes g_1, g_1 \otimes g_2)$. We saw that $g_1 \otimes g_1(0) = (1-\delta_1)^2 > (1-\delta_1)(1-\delta_2) = g_1 \otimes g_2(0)$, so $g_1 \otimes g_2$ starts lower. Since $-e^{2\epsilon_1} < -e^{\epsilon_1 + \epsilon_2} < -e^{\epsilon_1 - \epsilon_2}$, $g_1 \otimes g_1 < -1$, we know that $g_1 \otimes g_1$ starts with a steepest slope, which stops first (at $\alpha_1^{\star2} < \alpha_1^{\star}\alpha_2^{\star}$) and then gets the smallest second slope (until at least $\alpha \geq \alpha_{g_1 \otimes g_2}^{\star}$).
    As a result, either that first slope is steep enough such that $g_1 \otimes g_1(\alpha_1^{\star2}) < g_1 \otimes g_2(\alpha_1^{\star2})$ and the composed curves cross, or they will never cross and $g_1 \otimes g_2 \leq g_1 \otimes g_1$.
    %
    We treat each case in turn.

    \emph{Case 1:} $g_1 \otimes g_1(\alpha_1^{\star2}) < g_1 \otimes g_2(\alpha_1^{\star2})$ and the composed curves cross (\Cref{subfig:counter-example-crossing}).
    In that case, we know that $\conv(g_1 \otimes g_1, g_1 \otimes g_2)(\alpha_1^{\star2}) = g_1 \otimes g_1(\alpha_1^{\star2}) = (1-\delta_1)^2 - e^{2\epsilon_1} \alpha_1^{\star2}$.
    As a result, the slope of $\conv(g_1 \otimes g_1, g_1 \otimes g_2)$ in $\alpha \in [0, \alpha_1^{\star2}]$ is $\frac{(1-\delta_1)^2 - e^{2\epsilon_1} \alpha_1^{\star2} - (1-\delta_1)(1-\delta_2)}{\alpha_1^{\star2}} = \frac{(1-\delta_1)(\delta_2 - \delta_1)}{\alpha_1^{\star2}} - e^{2\epsilon_1} = (1+e^{\epsilon_1})\frac{\delta_2 - \delta_1}{\alpha_1^{\star}} - e^{2\epsilon_1}$ (the last step uses $\alpha_i^\star = \frac{1 - \delta_i}{1 + e^{\epsilon_i}}$).
    
    Remember that $g_1 \otimes \conv(g_1, g_2)(0) = (1-\delta_1)(1-\delta_2) = \conv(g_1 \otimes g_1, g_1 \otimes g_2)$, and the first slope of with first slope of $g_1 \otimes \conv(g_1, g_2) = \frac{\delta_2 - \delta_1}{\alpha_1^\star}e^{\epsilon_1} - e^{2\epsilon_1} < (1+e^{\epsilon_1})\frac{\delta_2 - \delta_1}{\alpha_1^{\star}} - e^{2\epsilon_1}$ since $\delta_2 - \delta_1 > 0$.

    We have shown that in this case, $g_1 \otimes \conv(g_1, g_2) > \conv(g_1 \otimes g_1, g_1 \otimes g_2)$ at least on $(0, \alpha_1^{\star2}]$, so by \Cref{thm:universal_free_natural_filter}, and there are no $f$-DP filters.

    % Choosing $f=$
    
    \emph{Case 2:} $g_1 \otimes g_1(\alpha_1^{\star2}) \geq g_1 \otimes g_2(\alpha_1^{\star2})$ and $\conv(g_1 \otimes g_1, g_1 \otimes g_2) = g_1 \otimes g_2$ (\Cref{subfig:counter-example-not-crossing}).
    %
    Intuitively, we know that the first slope of $\conv(g_1, g_2)$ is steeper than that of $g_2$, since $g_2$ starts lower and crosses with $g_1$. Formally, the first slope of $g_1 \otimes \conv(g_1, g_2)$ is $\frac{\delta_2 - \delta_1}{\alpha_1^\star}e^{\epsilon_1} - e^{2\epsilon_1} = - e^{\epsilon_1} \big( e^{\epsilon_1} - \frac{\delta_2 - \delta_1}{\alpha_1^\star} \big)$. We know that $1-\delta_1-e^{\epsilon_1}\alpha_1^{\star} = g_1(\alpha_1^{\star}) < g_2(\alpha_1^{\star2}) = 1-\delta_2-e^{\epsilon_2}\alpha_1^{\star}$, so $\delta_2-\delta_1 < \alpha_1^{\star}(e^{\epsilon_1} - e^{\epsilon_2})$ and $0 < e^{\epsilon_2} < e^{\epsilon_1} - \frac{\delta_2-\delta_1}{\alpha_1^{\star}}$.
    That is, $-e^{\epsilon_1} \big( e^{\epsilon_1} - \frac{\delta_2 - \delta_1}{\alpha_1^\star} \big) < -e^{\epsilon_1}e^{\epsilon_2}$.

    In this case as well, the first slope of $g_1 \otimes \conv(g_1, g_2)$ is strictly more steep than that of $\conv(g_1 \otimes g_1, g_1 \otimes g_2)$, so $g_1 \otimes \conv(g_1, g_2) > \conv(g_1 \otimes g_1, g_1 \otimes g_2)$ at least on $(0, \alpha_1^{\star2}]$. By \Cref{thm:universal_free_natural_filter}, and there are no free $f$-DP filters.
\end{proof}

% \newpage
% 



\section{Old prelims}
\label{app:old_prelim}

\subsection{Background on \texorpdfstring{$f$}{f}-DP}
\label{sec:background}

One way to formalize ``is hard to distinguish'' could be done using hypothesis testing. 
For two neighbouring datasets $\dataset_0, \dataset_1$: 
\begin{equation*}
   \hypothesis_0:  \text{the dataset is $\dataset_0$} \quad 
\text{versus} \quad 
  \hypothesis_1: \text{the dataset is $\dataset_1$},
\end{equation*}
where the output of the mechanism is the basis for hypothesis testing. 

In other words, an attacker is given $\mechanism(\dataset)$ for $\dataset \in \{\dataset_0, \dataset_1\}$
and need to reject either $\hypothesis_0$ or $\hypothesis_1$.
The type I error in this case is incorrectly rejecting $\hypothesis_0$ {\color{blue}given that $D = D_0$} and type II error is incorrectly
rejecting $\hypothesis_1$ {\color{blue}given that $D = D_1$}.
We say that $\mechanism$ is $f$-DP if and only if no attacker can simultaneously bound the probability of type I error
by $\alpha$ and the probability of type II error by $f(\alpha)$. {\color{blue}(Bingshan: upper bound type I error by $\alpha$ and upper bound type II error by $f(\alpha)$? Why we need to have the wording ``probability''? )}

In other words, let us assume $P$ and $Q$ are two distributions over $\mathcal{R}$;
a rejection rule $0 \le \phi \le 1$ {\color{blue}(Bingshan: how about using $\phi : \mathcal{R} \rightarrow [0,1]$? anyway, we need to use this version when applying Nerman-Pearson lemma)}, with type I and type II error rates defined as
$\alpha_\phi = \E_{x \sim P}{\phi(x)}$ and $\beta_\phi = 1 - \E_{x \sim Q}{\phi(x)}$.

\begin{lemma}[Neyman-Pearson lemma] Let $P$ and $Q$ be probability distributions on $\mathcal{R}$ with densities $p$ and $q$, respectively. For the hypothesis testing problem $H_0: P \text{ versus } H_1: Q$, a test $\phi : \mathcal{R} \rightarrow [0,1]$ is the most powerful test at level $\alpha$ if and only if there exist two constants\footnote{These two constants can depend on $\alpha$ and distribution-dependent parameters.} $h \in [0, + \infty)$ and $c \in [0,1]$ such that $\phi$ has the form
\begin{equation}
     \phi(x) = \begin{cases}
      1, & \text{if } p(x) < h q(x),\\
      c, & \text{if }  p(x) = hq(x),\\
      0, & \text{if } p(x)> hq(x),
    \end{cases} 
\end{equation}
and $\mathbb{E}_{X \sim P}[\phi(X)] = \alpha$.
    
\end{lemma}

We say that $P$ and $Q$ are $f$-DP indistinguishable if and only if $\beta_\phi \ge f(\alpha_\phi)$ for any $\phi$. 
This leads us to a notion of a tradeoff curve.
\begin{definition}
    Let $P$ and $Q$ be some distributions over a set $\mathcal{R}$. 
    The tradeoff curve $T(P, Q) : [0, 1] \to [0, 1]$ is defined as follows: 
    $T(P, Q)(\alpha) = \inf \{\beta_\phi ~:~ \alpha_\phi \le \alpha\}$,
    where the infimum is over all rejection rules $\phi$.
\end{definition}

\begin{proposition}[Proposition 2.2 \citep{DRS22}] A tradeoff curve is convex, continuous, non-increasing, and  symmetric with respect to the line $y=x$. 
    
\end{proposition}



%ADD proposition of tradeoff curves. tradeoff curves T (·, ·), and convex, continuous, symmetric w.r.t y = x. Let T be the set of all tradeoff curves. 


    




A randomized mechanism $\mathcal{M}$ induces a distribution over outputs $P_0 \equiv \mathcal{M}(D_0)$ or $P_1 \equiv \mathcal{M}(D_1)$, when running on $D_0$ or $D_1$, respectively.
\begin{definition}[$f$-differential privacy] Let $f$ be a tradeoff curve. A mechanism $\mathcal{M}$ is said to be $f$-differentially private if 
\begin{equation*}
    T(\mathcal{M}(D_0), \mathcal{M}(D_1)) \ge f,
\end{equation*}
for all neighbouring datasets $D_0$ and $D_1$.
\end{definition}


 
Like other DP notions, the notion of $f$-DP considers any pair of neighbouring datasets $D_0$ and $D_1$, that differ in only one element (or add/remove it doesn't matter). 
%{\color{blue} (tradeoff curves for identity, approx DP $f_{\epsilon, \delta}$, GDP $G_\mu$, .)}
%To cover:
%\begin{itemize}
    %\item Rejection rules.
    %\item power, level.
    %\item tradeoff curves $T(\cdot, \cdot)$, and convex, continuous, symmetric w.r.t $y=x$. 
%  \item $T(\cdot, \cdot)$ can be realized. construction for distributions realizing tradeoff functions from \cite{DRS22}, Proposition 2.2.
   % \item tradeoff curves for identity, approx DP $f_{\epsilon, \delta}$, GDP $G_\mu$, approx GDP ($f_{0, \delta} \otimes G_\mu$).
  %  \item Composition (+ useful properties such as: composition does not depend on the realizer for $f_1$ $f_2$ and $f_1 \otimes f_2 = T(P_0 \times Q_0, P_1 \times Q_1)$ for any realizer).
   % \item Filters (\Cref{alg:privacy-filter})
%\end{itemize}
Let $\cT$ be the set of all tradeoff curves.
 $\fid \triangleq \alpha \rightarrow 1-\alpha$ is the identity tradeoff curve ($\forall f, \ f \otimes \fid = f$) corresponding to perfect privacy. 
A tradeoff curve $f$ is always $\leq \fid$. $f$-DP is a generalization of the classic $(\epsilon, \delta)$-DP.
\begin{proposition}
    A mechanism $\mathcal{M}$ is $(\epsilon, \delta)$-DP if and only if 
$\mathcal{M}$ is $f_{\epsilon, \delta}$-DP, where 
$f_{\epsilon, \delta}(\alpha) = \max \left\{0,1-\delta-e^{\epsilon}\alpha, e^{-\epsilon}(1-\delta-\alpha) \right\}$.    
\end{proposition}

\begin{definition}
    A mechanism $\mathcal{M}$ is said to be $\mu$-GDP if
    \begin{equation*}
        T(\mathcal{M}(D_0), \mathcal{M}(D_1)) \ge T(\mathcal{N}(0,1), \mathcal{N}(\mu, 1)),
    \end{equation*}
    for all neighbouring datasets $D_0$ and $D_1$.
\end{definition}
Let $G_{\mu} := T(\mathcal{N}(0,1), \mathcal{N}(\mu, 1))$. Then, we have $G_{\mu}(\alpha) = \Phi(\Phi^{-1}(1-\alpha)-\mu)$.
%{\color{magenta}(Bingshan: perfect DP means $(0,0)$-DP.)}



\begin{proposition}[Proposition~2.2 \citep{dong2022gaussian}]\label{prop:tf-distr} For any tradeoff curve $f \in \mathcal{T}$, there exist a pair of distributions $P, Q$ over a common measurable space such that $f = T(P, Q)$.
\end{proposition}
% \begin{proposition}[Proposition~2.2 \citep{dong2022gaussian}]\label{prop:tf-distr}
%     \[
%         \forall f \in \cT, \ \exists P, Q \ \textrm{s.t.} \ f = T(P, Q).
%     \] 
% \end{proposition}
\begin{proof}
   The common measurable space is the unit interval $[0,1]$, and the common measure $\nu$ is the uniform over $[0,1]$ with a point-mass at $1$.
   Let $P$ be the uniform distribution over $[0,1]$. Let  $Q$ be a distribution over $[0,1]$ having the following density function:
\begin{equation}
     q(x) = \begin{cases}
      -f'(1-x), & \text{if } x \in [0,1) \ \text{and} \ f' \ \text{exists},\\
      (1-f(0)) \cdot \delta(x-1), & \text{if }  x = 1.
    \end{cases} 
    \label{eq: density}
\end{equation}
Since $f$ is convex, $f'$ exists almost everywhere $\int_{0}^1 q(x) d\nu(x)  =\int_{0}^1 (-f'(1-x)) d\nu(x) + 1 - f(0) = f(0)-f(1)+1-f(0) = 1$, the function shown in  (\ref{eq: density}) is indeed a probability density function. Without loss of generality, we assume that all $-f'(1-x)$ are distinct.
Now, we construct  a test $P \text{ vs } Q$.
For any $\alpha \in [0,1]$,
%we let $1 = h \cdot q(1-\alpha)$. 
the optimal rejection rule is
\begin{equation}
     \phi(x) = \begin{cases}
      1, & \text{if } x \ge 1-\alpha ,\\
      0, & \text{if } x < 1-\alpha.
    \end{cases} 
\end{equation}
Then, we have $\mathbb{E}_{P}[\phi] = \alpha$ and $T(P,Q)(\alpha) = 1- \mathbb{E}_{Q}[\phi]  = 1- \left(\int_{1-\alpha}^{1} (-f'(1-x)) dx + 1 - f(0)\right) = f(\alpha)$, which concludes the proof. \end{proof}




\subsection{Background on composition and differential privacy filters}

\todo{maybe something saying that strong composition is implied by $f$-DP composition so when we have free filters we also have free strong composition filters. Add something similar for privacy loss distribution?}

\begin{enumerate}
    \item Worst-case composition (basic composition): Composition (+ useful properties such as: composition does not depend on the realizer for $f_1$ $f_2$ and $f_1 \otimes f_2 = T(P_0 \times Q_0, P_1 \times Q_1)$ for any realizer). Add $(\epsilon, \delta)$-DP composition and GDP composition. approx GDP ($f_{0, \delta} \otimes G_\mu$)

    \item Adaptive composition: Filters (\Cref{alg:privacy-filter}). ADD GDP adaptive composition. 


\end{enumerate}
\subsection{Notations, definitions, and useful properties}
\label{sec:background-results}

Given a set of tradeoff curves $\cF$, we will note $\cF^k = \{ f_1 \otimes \cdots \otimes f_k : f_1, \ldots, f_k \in \cF \}$ the set of tradeoff curves we obtain under $k$ compositions of functions in $\cF$. We note $\cF^\infty = \lim_{k\rightarrow\infty} \cF^k$. Notice that as long as $\fid \in \cF$, we have $\cF^k \subseteq \cF^{k+1}$.
\ml{I might be making the assumption $\fid \in \cF$ in some places, hence the note below to maybe say $\cF \cup \{f_{\textrm{id}}\}$. Maybe we should add it (if useful). Leaving this not to forget.}
Our study relies on properties of the lower-convex envelope of a set of tradeoff curves:

\begin{definition}[Lower-convex envelope]
    We call $\conv$ the lower-convex envelope operator. That is, for any set of tradeoff curves $\cF$, we have:
    \[
        \conv(\cF)(\alpha) = \sup \big\{ g(\alpha) : \textrm{g is convex and } \forall f \in \cF, \ g \leq f \big\}
    \]
\end{definition}
% {\color{red}(Bingshan: the above definition is so weird. How about this way?
% \begin{equation}
%     \conv(\cF) = \sup \big\{ g : \textrm{g is convex and } \forall f \in \cF, \forall \alpha \in [0,1], \ g(\alpha) \leq f(\alpha) \big\}
% \end{equation}
% )}

{\color{red}This lower-convex envelope has XXX properties that will be useful to prove our main results.}
\begin{proposition}\label{prop:convex-bridge-two-point-support}
For any set of tradeoff curves $\cF$, for all $\alpha \in [0, 1]$, there exists $f_1, f_2 \in \cF$, $x_1, x_2 \in [0,1]$, and $w \in [0, 1]$ such that:
\[
    \conv(\cF)(\alpha) = w f_1(x_1) + (1-w) f_2(x_2) .
\]
\end{proposition}
\begin{proof}    \ml{TODO: should be straightforward but not sure how to write it concisely yet. Seemed like Carathéodory's theorem but not quite.}. {\color{red}(Bingshan: Page 74 in \cite{DRS22} ).}
\end{proof}


\begin{proposition}[One-step convex combination]
\label{prop:one-step-convex-combination}
Consider a set of tradeoff curves $\cF$, and an adversary $\cA$ that picks any $f\in\cF$ and receives $m \sim \cM$ for any $f$-DP $\cM$. $\cA$'s view is $\conv(\cF)$-DP.
\end{proposition}
\begin{proof}
    $\forall f \in \cF, \ f \geq \conv(\cF)$. 
    \ml{TODO:convex combination of $f$'s in the expectation and we're good}

%    {\color{blue}Bingshan: because $\conv(\cF)$ is always below any $f \in \mathcal{F}$.}
\end{proof}

{\color{blue}(Bingshan: here, I suggest adding a definition for the piecewise-linear tradeoff curves, i.e., a tradeoff curve that is made up of multiple, straight-line segments, where each segment is defined on a specific interval.) } The composition of tradeoff curves is challenging to compute in general, 
but piecewise-linear tradeoff curves (which include  all $(\epsilon, \delta)$-DP mechanisms) 
admit a closed form algorithm. 
\ml{
    This is obvious I think, but I haven't seen it in that form before. 
    Probably the same a the "optimal composition" papers. Maybe we can point this out and cite? \citep{murtagh2015complexity,kairouz2015composition}
}
This algorithm is not efficient 
\ml{
    in fact this is hard in the general case, same paper(s) as comment right above
}, 
but tractable in simple cases. 
We will leverage this result for illustrative numerical examples, to show counter-examples, 
and in a result on $(\epsilon, \delta)$-DP.


